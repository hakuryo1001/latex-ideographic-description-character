\documentclass[a5paper, 10pt, openany]{book} % A5 paper size

% \usepackage[paperwidth=148mm, paperheight=210mm, top=1.27cm, bottom=1.27cm, inner=1.9cm, outer=1.27cm, headsep=1.5cm, footskip=1.75cm]{geometry} % Custom dimensions and margins
% Adjusted margins to ensure space for page numbers
\usepackage[paperwidth=148mm, paperheight=210mm, 
            top=1.27cm, bottom=2.5cm, inner=1.9cm, outer=1.27cm, 
            headsep=1.5cm, footskip=1.5cm]{geometry}


\usepackage[utf8]{inputenc}
\usepackage{ctex}

% *-----------------------------------------------------------------------*
% | Fonts and typography                                                  |
% *-----------------------------------------------------------------------*

% Set CJK main font (for Chinese/Japanese/Korean characters)

\setmainfont{Times New Roman}
\setCJKmainfont{BabelStone Han}
% \setCJKmainfont{JyutcitziWithSourceHanSerifTCLight}



% You can also use \newfontfamily for custom non-CJK fonts if needed
% \setCJKmainfont{JyutcitziWithPMingLiURegular}[Path = ./, Extension = .ttf]
% \setCJKmainfont{JyutcitziWithSourceHanSerifTCRegular}[Path = ./, Extension = .ttf]




% \newfontfamily{\jcz}{JyutcitziWithPMingLiURegular}[Path = /Users/hongjan/Library/Fonts/, Extension = .ttf]
% \newfontfamily{\jcz}{JyutcitziWithSourceHanSansHCRegular}[Path = /Users/hongjan/Library/Fonts/, Extension = .ttf]
% This has the best rendition for latin characters 
\newfontfamily{\jcz}{JyutcitziWithSourceHanSerifTCLight}[Path = /Users/hongjan/Library/Fonts/, Extension = .ttf]



% *-----------------------------------------------------------------------*
% | Math & Equations     |
% *-----------------------------------------------------------------------*
\usepackage{amsmath} % For advanced math formatting
\usepackage{amssymb} % For mathematical symbols
\usepackage{tikz} % For drawing logic decision trees



% *-----------------------------------------------------------------------*
% | Table Management                                                      |
% *-----------------------------------------------------------------------*

 

\usepackage{graphicx}
\usepackage{array}
\usepackage{tabularx}
\usepackage[table,xcdraw]{xcolor}
% 

% Load ruby package for furigana (Ruby text)
\usepackage{ruby}


% *-----------------------------------------------------------------------*
% | Chinese and Soochow Numerals for chapter management                   |
% *-----------------------------------------------------------------------*

% Define Chinese numerals for numbers 1-99
% 〇〡〢 〣 〤 〥 〦 〧 〨 〩 十 〹 〺 卅

\newcommand{\soochowNumeral}[1]{%
  \ifnum#1<10
    \ifcase#1 〇\or 〡\or 〢\or 〣\or 〤\or 〥\or 〦\or 〧\or 〨\or 〩\fi%
  \else
    \ifnum#1<20
      〸\soochowUnits{\numexpr#1-10\relax}%
    \else
      \ifnum#1<30
        〹\soochowUnits{\numexpr#1-20\relax}%
      \else
        \ifnum#1<40
          〺\soochowUnits{\numexpr#1-30\relax}%
        \else
          \ifnum#1<50
            卅\soochowUnits{\numexpr#1-40\relax}%
          \else
            \ifnum#1<60
              〥十\soochowUnits{\numexpr#1-50\relax}%
            \else
              \ifnum#1<70
                〦十\soochowUnits{\numexpr#1-60\relax}%
              \else
                \ifnum#1<80
                  〧十\soochowUnits{\numexpr#1-70\relax}%
                \else
                  \ifnum#1<90
                    〨十\soochowUnits{\numexpr#1-80\relax}%
                  \else
                    \ifnum#1<100
                      〩十\soochowUnits{\numexpr#1-90\relax}%
                    \fi
                  \fi
                \fi
              \fi
            \fi
          \fi
        \fi
      \fi
    \fi
  \fi
}

% Helper command for units (ones digit)
\newcommand{\soochowUnits}[1]{%
  \ifnum#1=0
  \else
    \ifnum#1<4
      \ifcase#1 \or 一\or 二\or 三\fi%
    \else
      \soochowNumeral{#1} % Use Suzhou numeral for numbers greater than 3
    \fi
  \fi
}

\newcommand{\chinesenumeral}[1]{%
  \ifnum#1<10
    \ifcase#1 〇\or 一\or 二\or 三\or 四\or 五\or 六\or 七\or 八\or 九\fi%
  \else
    \ifnum#1<20
      十\chinesenumeral{\numexpr#1-10\relax}%
    \else
      \ifnum#1<30
        二十\chinesenumeral{\numexpr#1-20\relax}%
      \else
        \ifnum#1<40
          三十\chinesenumeral{\numexpr#1-30\relax}%
        \else
          \ifnum#1<50
            四十\chinesenumeral{\numexpr#1-40\relax}%
          \else
            \ifnum#1<60
              五十\chinesenumeral{\numexpr#1-50\relax}%
            \else
              \ifnum#1<70
                六十\chinesenumeral{\numexpr#1-60\relax}%
              \else
                \ifnum#1<80
                  七十\chinesenumeral{\numexpr#1-70\relax}%
                \else
                  \ifnum#1<90
                    八十\chinesenumeral{\numexpr#1-80\relax}%
                  \else
                    \ifnum#1<100
                      九十\chinesenumeral{\numexpr#1-90\relax}%
                    \fi
                  \fi
                \fi
              \fi
            \fi
          \fi
        \fi
      \fi
    \fi
  \fi
}

% Custom chapter title formatting with Chinese numeral chapter numbers
\usepackage{titlesec}


% Custom chapter title formatting with Chinese numeral chapter numbers
\titleformat{\chapter}[block] % 'block' means the title appears on a new line
  {\Huge\bfseries} % Font size and bold formatting for the title
  {\soochowNumeral{\thechapter}} % Chinese character for chapter number
  {1em} % Space between the number and the title
  {\Huge} % Custom style for the chapter title itself (can modify)

% Remove the default LaTeX behavior of forcing new chapters to start on a new page
\makeatletter
\renewcommand\chapter{\if@openright\cleardoublepage\else\clearpage\fi
  \thispagestyle{plain}%
  \global\@topnum\z@
  \@afterindentfalse
  \secdef\@chapter\@schapter}
\makeatother



% *-----------------------------------------------------------------------*
% | Proof trees                                                              |
% *-----------------------------------------------------------------------*
\usepackage[tableaux]{prooftrees}
\renewcommand*\linenumberstyle[1]{(#1)}
\RequirePackage{mdwtab,latexsym,amsmath,amsfonts,ifthen}

%Line height in proofs
\newlength{\fitchlineht}
\setlength{\fitchlineht}{1.5\baselineskip}
% Horizontal indent between proof levels
\newlength{\fitchindent}
\setlength{\fitchindent}{0.7em}
% Indent to comment
\newlength{\fitchcomind}
\setlength{\fitchcomind}{2em}
% Line number width
\newlength{\fitchnumwd}
\setlength{\fitchnumwd}{1em}

% Altered from mdwtab.sty: shorter vline, for start of subproof
\makeatletter
\newcommand\fvline[1][\arrayrulewidth]{\vrule\@height.5\fitchlineht\@width#1\relax}
\makeatother
% Ordinary vertical line
\newcommand{\fa}{\vline\hspace*{\fitchindent}}
% Vertical line, shorter: Use at start of (sub)proof
\newcommand{\fb}{\fvline\hspace*{\fitchindent}}
% Hypothesis
\newcommand{\fh}{\fvline%
  \makebox[0pt][l]{{%
      \raisebox{-1.4ex}[0pt][0pt]{\rule{1.5em}{\arrayrulewidth}}}}%
  \hspace*{\fitchindent}}
% Hypothesis, with longer vert line: for >1 hypothesis
\newcommand{\fj}{\vline%
  \makebox[0pt][l]{{%
      \raisebox{-1.4ex}[0pt][0pt]{\rule{1.5em}{\arrayrulewidth}}}}%
  \hspace*{\fitchindent}}
% Modal subproof: takes argument = operator
\newcommand{\fitchmodal}[1]{% 
  \makebox[0pt][r]{${}^{#1}$\,}\fvline\hspace*{\fitchindent}}
\newcommand{\fn}{\fitchmodal{\Box}}% Box subproof 
\newcommand{\fp}{\fitchmodal{\Diamond}}% Diamond subproof
% Modal subproof with hypothesis in first line (as in Fitch)
\newcommand{\fitchmodalh}[1]{% 
  \makebox[0pt][r]{${}^{#1}$\,}%
  \fvline%
  \makebox[0pt][l]{{%
      \raisebox{-1.4ex}[0pt][0pt]{\rule{1.5em}{\arrayrulewidth}}}}%
  \hspace*{\fitchindent}}
% Rule: formula introduction marker. \fr with line, \fs without line
\newcommand{\fr}{%
  \makebox[0pt][r]{${\rhd}$\,\,}\vline\hspace*{\fitchindent}}
\newcommand{\fs}{%
  \makebox[0pt][r]{${\rhd}$\,\,}}
% Box around argument, like new variable in ql
\newcommand{\fw}[1]{\fbox{\footnotesize $#1$}}

% 
\newcounter{fitchcounter}
\setcounter{fitchcounter}{0}
%To avoid starting from 1, \setboolean{resetfitchcounter}{false}
\newboolean{resetfitchcounter}
\setboolean{resetfitchcounter}{true}
%To avoid increasing numbers, \setboolean{increasefitchcounter}{false}
\newboolean{increasefitchcounter}
\setboolean{increasefitchcounter}{true}
%\formatfitchcounter can be altered if need be, though only once per proof
\newcommand{\formatfitchcounter}[1]{\scriptsize \arabic{#1}}
%Typeset the counter
\newcommand{\fitchcounter}{%
  \ifthenelse{\boolean{increasefitchcounter}}{\addtocounter{fitchcounter}{1}}{}
  \formatfitchcounter{fitchcounter}}

%A line with a special number -- a tag, e.g. \ftag{\vdots}{}
\newcommand{\ftag}[2]{\multicolumn{1}%
  {!{\makebox[\fitchnumwd][r]{#1}\hspace{\fitchindent}}Ml@{\hspace{\fitchcomind}}}%
  {#2}}

\newenvironment{fitchnum}%
{\ifthenelse{\boolean{resetfitchcounter}}{\setcounter{fitchcounter}{0}}{}
  \begin{tabular}{!{\makebox[\fitchnumwd][r]{\fitchcounter }\hspace{\fitchindent}}Ml@{\hspace{\fitchcomind}}l}}%
{\end{tabular}}

\newenvironment{fitchunum}%
{\begin{tabular}{!{\makebox[\fitchnumwd][r]{}\hspace{\fitchindent}}Ml@{\hspace{\fitchcomind}}l}}%
{\end{tabular}}

\newenvironment{fitch}{\renewcommand{\arraystretch}{1.5}
  \begin{fitchnum}}{\end{fitchnum}}
\newenvironment{fitch*}{\renewcommand{\arraystretch}{1.5}
  \begin{fitchunum}}{\end{fitchunum}}

% The following is useful for giving a numbered formula, then the proof.
\newenvironment{flem}[2]%
{\begin{eqnarray}
    &#1\label{#2}\\
    &\begin{fitch}}%
    {\end{fitch}\notag\end{eqnarray}}

%To write comment field for two consecutive lines, with brace
\newcommand{\ftwocom}[1]{%
  \parbox[t]{3cm}{
    \raisebox{-.6\baselineskip}[\baselineskip][0pt]{%
      $\left.
        \begin{aligned}
          \,\\ \,
        \end{aligned}
      \right\}$\quad #1}
  }}

\usepackage{amssymb,amsmath}
\usepackage{amsthm}
\setlength{\parindent}{0ex}
\newtheorem{theorem}{Theorem}[section]
\newtheorem{corollary}{Corollary}[theorem]
\newtheorem{lemma}{Lemma}
\newtheorem{definition}{Definition}
\newtheorem{example}{Example}
\usepackage{adjustbox}
% \setlength{\parskip}{0.5em}
\usepackage{multirow}
\usepackage{booktabs}


\usepackage{array}
\begin{document}
% to avoid overfull hbox
\sloppy
% % \jcz{} must be run so the document can process jyutcitzi 
\jcz{} 

% % \jczSourceHan{}




\chapter{第一章}

目錄\\《自序》	3\\《橫紋柴》	4\\《七畝肥田》	39\\《邱瓊山》	45\\《積福兒郎》	50\\《閃山風》	56\\《九魔托世》	63\\《饑荒詩》	69\\《瓜棚遇鬼》	70\\《鬼怕孝心人》	73\\《張閻王》	75\\《修整爛命》	79\\《骨肉試真情》	85\\《潑婦》	98\\《生魂遊地獄》	108\\《借火食烟》	122\\《好秀才》	126\\《砒霜砵》	159\\《茅寮訓子》	170\\\\ 


光緒丙申年新鎸,邵紀棠先生輯,羊城太平新街以文堂藏板。

% \jcz{} \\

󱑘󱑙󱑚󱑛󱑜󱑝



  󱍑  󰼐
  󱍑  󰼐
  󱍑  󰼐
  󱍑  󰼐
々
\chapter{《自序》}

% Example of furigana over kanji
これは\ruby{漢字}{かんじ}の例です。 %漢字 will display "かんじ" as furigana above it

% You can also use furigana for names or specific terms
\section{漢字と\ruby{平仮名}{ひらがな}}

この文章は、\ruby{日本語}{にほんご}を練習するためのサンプルです。

The quick brown fox jumps over the lazy dog.The quick brown fox jumps over the lazy dog.

󱜩
The quick brown fox jumps over the lazy dog.The quick brown fox jumps over the lazy dog.

\ruby{}{而家}搞$^{'}$
  󱍑  󰼐
󱑡


\ 
朕惟フニ我カ皇祖皇宗國ヲ肇ムルコト宏遠ニ德ヲ樹ツルコト深厚ナリ我カ臣民克ク忠ニ克ク孝ニ億兆心ヲ一ニシテ世世厥ノ美ヲ濟セルハ此レ我カ國體ノ精華ニシテ敎育ノ淵源亦實ニ此ニ存ス爾臣民父母ニ孝ニ兄弟ニ友ニ夫婦相和シ朋友相信シ恭儉己レヲ持シ博愛衆ニ及ホシ學ヲ修メ業ヲ習ヒ以テ智能ヲ啓發シ德器ヲ成就シ進テ公益ヲ廣メ世務ヲ開キ常ニ國憲ヲ重シ國法ニ遵ヒ一旦緩急アレハ義勇公ニ奉シ以テ天壤無窮ノ皇運ヲ扶翼スヘシ是ノ如キハ獨リ朕カ忠良ノ臣民タルノミナラス又以テ爾祖先ノ遺風ヲ顯彰スルニ足ラン斯ノ道ハ實ニ我カ皇祖皇宗ノ遺訓ニシテ子孫臣民ノ俱ニ遵守スヘキ所之ヲ古今ニ通シテ謬ラス之ヲ中外ニ施シテ悖ラス朕爾臣民ト俱ニ拳々服膺シテ咸其德ヲ一ニセンコトヲ庶幾フ
以呂波耳本部止
千利奴流乎和加
餘多連曽津祢那
良牟有為能於久
耶万計不己衣天
阿佐伎喩女美之
恵比毛勢須


諸行無常\\
是生滅法\\
生滅滅已\\
寂滅為楽\\

Shogyō mujō
Zeshō meppō
Shōmetsu metsui
Jakumetsu iraku

いろはにほへと	Iro fa nifofeto	色は匂えど	Iro wa nioedo	1–7	Even the blossoming flowers [Colors are fragrant, but they]
ちりぬるを	Tirinuru wo	散りぬるを	Chirinuru o	8–12	Will eventually scatter
わかよたれそ	Wa ka yo tare so	我が世誰ぞ	Wa ga yo tare zo	13–18	Who in our world
つねならむ	Tune naramu	常ならん	Tsune naran	19–23	Shall always be? (= つねなろう)
うゐのおくやま	Uwi no okuyama	有為の奥山	Ui no okuyama	24–30	The deep mountains of karma—
けふこえて	Kefu koyete	今日越えて	Kyō koete	31–35	We cross them today
あさきゆめみし	Asaki yume misi	浅き夢見じ	Asaki yume miji	36–42	And we shall not have superficial dreams
ゑひもせす	Wefi mo sesu	酔いもせず	Ei mo sezu¹
Yoi mo sezu	43–47	Nor be deluded.


\section{Background}
This is the first section in the chapter.

\subsection{History}
This is the subsection under "Background."

\subsubsection{Ancient History}
This is a subsubsection under "History."

\paragraph{Key Events}
This is a paragraph under "Ancient History."

\subparagraph{Event Details}
This is a subparagraph under "Key Events."


語云:知多世事胸襟濶,識透人情眼界寬。知識兩字,由於自己之想象而明,亦由聞人之談論而得也。嘗見街頭巷尾月下燈前,閒坐成群,未嘗無語,但所論多無緊要之事,未足以有補身心。或有談及因果報應,則有聽有不聽焉,且有抽身而去者矣。非言語不通,實事情未得趣也。惟講得有趣,方能入人耳、動人心,而留人餘步矣。善打鼓者,多打鼓邊;善講古者,須談別致。講得深奧,婦孺難知,惟以俗情俗語之說通之,而人皆易曉矣,且津津有味矣。誦讀之暇,採古事數則,有時說起,聽者忘疲。因付之梓人,以備世之好言趣致者。\\ 
語云:知多世事胸襟濶,識透人情眼界寬。知識兩字,由於自己之想象而明,亦由聞人之談論而得也。嘗見街頭巷尾月下燈前,閒坐成群,未嘗無語,但所論多無緊要之事,未足以有補身心。或有談及因果報應,則有聽有不聽焉,且有抽身而去者矣。非言語不通,實事情未得趣也。惟講得有趣,方能入人耳、動人心,而留人餘步矣。善打鼓者,多打鼓邊;善講古者,須談別致。講得深奧,婦孺難知,惟以俗情俗語之說通之,而人皆易曉矣,且津津有味矣。誦讀之暇,採古事數則,有時說起,聽者忘疲。因付之梓人,以備世之好言趣致者。\\ 








\begin{table}[htbp]
  \jcz{}
  \centering
  \renewcommand{\arraystretch}{1.5} % Adjust row height
  \setlength{\tabcolsep}{4pt} % Adjust column padding
  \resizebox{\textwidth}{!}{
  \begin{tabularx}{\textwidth}{|X|X|X|X|}
  \hline
  % \rowcolor[HTML]{D0D0D0} 
  \textbf{坊間漢羅混用} & \textbf{漢字已整理版本} & \textbf{漢字粵切字混用(未組裝)} & \textbf{漢字粵切字混用(已組裝)} \\
  \hline
  咁都係果D嘢嘎啦,廿鯪蚊個餐又湯又剩唔通有得你食天九翅咩?求求其其有D肉有D菜蛋白質澱粉質撈撈埋埋打個白汁茄汁黑椒汁咁撐得你懵口懵面咪Lui返去返工返學返廠返寫字樓囉。唔係你估真係搵餐晏仔咁簡單啊。咁跟飯定跟意粉啊? 
  & 咁都係果啲嘢㗎啦,廿鯪蚊個餐又湯又剩唔通有得你食天九翅咩?求求其其有啲肉有啲菜蛋白質澱粉質撈撈埋埋打個白汁茄汁黑椒汁咁撐得你懵口懵面咪纍返去返工返學返廠返寫字樓囉。唔係你估真係搵餐晏仔咁簡單啊。咁跟飯定跟意粉啊? 
  & 丩今´都係丩个´大子¯野丩乍`力乍`,廿力正⁼蚊個餐又湯又剩𠄡通有得你食天九翅文旡¯?求々其々有大子¯肉有大子¯菜蛋白質澱粉質撈々埋々打個白汁茄汁黑椒汁丩今´止生゙得你懵口懵面文兮`力句¯返去返工返學返廠返寫字樓力个¯。𠄡係你估真係搵餐晏仔丩今`簡單⺍乍⁼。丩今´跟飯定跟意粉⺍乍`?
  & 󱜩都係󱟡󰦠野󱛒󰿒,廿󰻃蚊個餐又湯又剩𠄡通有得你食天九翅󰗘?求々其々有󰦠肉有󰦠菜蛋白質澱粉質撈々埋々打個白汁茄汁黑椒汁󱜩󰿽得你懵口懵面󰖚󰾠返去返工返學返廠返寫字樓󰼠。𠄡係你估真係搵餐晏仔󱜪簡單󰀓。󱜩跟飯定跟意粉󰀒? \\
  \hline
  \end{tabularx}
  }
\end{table}
  


\chapter{《橫紋柴》}
󱞢,
康熙間,四川省重慶府有一個舉人,姓安名維程,為人和平,無甚過處。生二子,長名大成,次名二成。大成之性,生來孝友;二成之性,一片愚頑。兩兄弟同胞不同性。安維程年四十餘,一病身故,剩下二子。田園可以足用,不至飢寒。大成之母沈氏,稟性極偏,不循道理,隨意所發,以執拗為能。此等賤婦潑婦,不是家庭之福。鄰里婦女多鄙薄之,加其號曰「橫紋柴」,其人可想矣。\\\\	橫紋柴見大成年紀有二十歲,為之婚娶。其新婦姓鄭,名珊瑚,生得十分美貌,極有禮義,柔聲下氣,奉事家婆。每朝晨早,定必到家婆處問安,捧茶獻餅,少不免修飾顏容,威儀致敬。誰不知橫紋柴一向性情佻撻,見珊瑚美麗,自覺懷慚,遂大聲罵曰:「做新婦敬家婆,是平常事,你估好時興麼?何用支支整整、聲聲色色,辦得個樣嬌嬈,想來我處賣俏嗎?我當初做新婦時,重好色水過你十倍!唔估今日老得個樣醜態,減去三分。」\\\\	家婆教新婦,理宜話:「亞嫂你都算有禮,但係仔乸上頭,駛乜咁拘束呢?粗衣麻布,到來問候,便是規模,不用太為着意。」如此說話,方是教道後生。你話佢賣俏,唔通做新婦,向家婆處賣俏麼?此等家婆就是惡得無理,而且講到自己做新婦時好色水,更不成個家教。\\\\	珊瑚聽罷,低頭順受,不敢出聲。明早又奉茶餅問安,粧得雅淡潔淨,着件洗水藍衫,頭面不施脂粉。橫紋柴一見又發怒曰:「昨朝話一句,今朝敢就花唔戴、粉唔搽、新衫唔着,想來激惱我,你估我唔知你!估我唔知!」極似惡婆聲口。珊瑚又低頭無語,自怨不曉奉承。自後,踢着櫈仔,將珊瑚罵;鷄唔食米,將珊瑚罵。珊瑚去探外家,三日歸來,被罵了十日。大成見老母不悅,遂將珊瑚拷打,以順母心。打得冤枉呀!橫紋柴暫時安然,不久病氣復發,古怪離奇,無情無理。\\\\	咒罵既慣,如鴉片烟引一樣,引起之時,唔咒罵唔做得。又如發冷症,三日一囘,或兩日一次。所以發冷有鬼,咒罵亦有鬼。發冷之鬼,至怕胡椒;咒罵之鬼,至怕口向火燒。\\\\	一晚,不過因些小事不合意,便企在門口大罵一場。珊瑚捧張竹椅出來,請婆婆安坐。橫紋柴坐下,腰骨挨斜,手指天、脚拍地,罵不絕聲。珊瑚煲茶一碗,捧來請婆婆解渴。橫紋柴飲了,喉嚨既潤,氣更高、聲更响。罵到三更,聲漸低、力漸微、氣漸喘。就是狗吠得多氣力都倦。珊瑚跪下稟曰:「婆婆所教,媳婦盡得聽聞,今知改過咯。請婆婆囘牀安睡,免至在此受了生風,通夜呌肚痛。」橫紋柴曰:「我要罵!我要罵!拚音伴之唔睡,罵到天光。」罵到豪興勃勃,人睡靜後又有鬼來聽。珊瑚從旁啼哭,鄰里共來勸止。珊瑚點燈來引,扶住歸房安歇。整好被鋪蚊帳,移正枕頭,囑咐婆婆安睡而去。\\\\	明早,即到家婆處問候。看見家婆唔出得聲,睜開雙眼總冇神情,發亂頭搖,似死一樣。嚇得珊瑚魂不附體,奔告鄰里。老伯婆一齊來到,一見光景,呵呵大笑,話珊瑚曰:「你唔在慌,佢不過昨晚劈大個口,出得氣多,撞了生風,蠱住個肚,以至血脉不通,精神困倦。靜養三兩日,自然好咯。」珊瑚方明其故。即買防風、羌活、蘇梗、薄荷,以驅風邪。又買黨參、北耆,以補元氣。食了兩劑,僅能出得聲、食得飯。\\\\	橫紋柴要買豬肉煲湯,以潤腸肚。珊瑚從命,照樣奉承。誰知肚內尚有風痰,未能疎發得透,食了豬肉,謂之傷風夾膩,啞了喉嚨,十餘日不能出得一語。請一個醫家先生來看脉,誰知此位先生,係初學手,唔識脉理,思疑風熱傳裏,悞用大黃、樸硝,大劑濃煎。橫紋柴飲了,疴得眼核俱深,瀉到周身疲倦,不能起坐,面黃骨瘦,不似人形。更兼瀉壞元神,脾胃俱弱,以至飲食無味,日覺乾枯。\\\\	橫紋柴一肚鬱勃不平之氣,憎厭無定之情,妙得兩味大黃、樸硝,瀉得乾乾淨淨,五腑六臟,忿恨皆消。此位先生精醫婦人惡毒,雖話初學工夫,其實可稱老手。\\\\	及後,另請過一個醫家,幾番調治,僅可開言。如是者有數月餘,頗見安靜。珊瑚暗中歡喜,以為婆婆納福,此後可以安枕無憂。誰知聲音响亮起來,仍係照前怒罵。大成出館讀書,身中常帶微病。橫紋柴罵珊瑚:「辦得好樣,致我個仔昏迷,傷損元氣。我個仔若死,要你命填償。」又罵大成不知好醜,唔中用,不顧身,貪愛老婆,致老母遇時憂慮。大成本來知得珊瑚賢孝,無奈老母不合意,遂寫分書一紙,吩咐珊瑚曰:「我聞娶妻所以事母,今致老母時時激惱,要妻何用。我將分書與你,你可別尋好處,另嫁他人,不宜在我屋住也。」話完翻袖出門而去。\\\\	珊瑚聞言,心神俱喪,將分書扯碎擲於火盤,歸房暗哭一夜。自知事不能挽,只得捲好袱包,擇三兩件緊用衣服,自行攜帶,其餘物件雖多,無心掛念也。拜別家堂香火,及沈氏婆婆,欲語不能成聲,濕洒兩行珠淚,垂頭喪氣,行步遲遲。出到門前,停足企住,想起當日出嫁之時,父兄叔伯戴纓帽、着長衫、點燈籠,一班隨護送我落轎,曾經囑咐教我孝順翁姑。今者被不孝之名,趕逐出來,有何面目歸家見父兄叔伯?不如一死便了。想完,即向袖裏拿出一張較剪仔,對正喉嚨,用力一剪。適值旁邊有一個婦人見他如此凶性,即用力擒住他手,盡勢推開,大喝一聲:「乜你咁勢凶呀!」誰知較剪已到喉處,僅傷喉皮,血出不止。此婦人即扯落珊瑚包頭帶,快快札住,大喊救命。鄰里紛紛走來,各拈跌打丸散來敷,止住血流。珊瑚挨凭門前,面如土色。各人看見,俱有可憐之意,或出嗟嘆之聲。\\\\	橫紋柴大罵曰:「你故意裝傷,想來累我。你要死,去歸外家處死,勿惹得咁多人在我門前嘈鬧。」旁人看見尚且悲傷,做了家婆,無一毫憐憫;大凡惡婆,良心先死。族中有一個守寡婦人,係王氏,素知珊瑚係好人。今家婆不容他在家,又既受傷不能行走,遂扶珊瑚歸到自己屋,買藥調理。不滿十日,傷痕好了。橫紋柴又來大罵曰:「你個賤人,既被丈夫逐出,為何不歸父母家?在此作我眼中釘,動我心頭火。」王氏答曰:「㗇㗇,你個橫紋柴,真正好笑咯!你個仔既寫分書,就如路人,那一個重係你新婦呀?走來罵人,問你醜唔醜?珊瑚係我親戚,我親戚來探,你都唔許佢住嗎?」罵得落花流水,無非代珊瑚出一肚悶氣。罵得橫紋柴無言可答,含羞忿忿,直走囘家。\\\\	珊瑚對王氏曰:「此處原非久住之所,我今去矣。」捲包袱直往姨婆家。姨婆嫁姓駱,即橫紋柴之大姐,大成之姨母也。年老而無夫,有媳守寡,而孫尚幼,與大成相離甚遠。平日來探,見珊瑚孝義,十分愛惜。故珊瑚投到其家,將事情略說與聽。姨婆曰:「我盡知我妹稟質奇離,不近人性,我是以懶於行探,為此故也。總之難為你受此抑屈淒涼。」珊瑚曰:「不關婆婆之事,總係我唔曉孝順,致激惱婆婆,自知罪該萬死。」只是怨自己不是,不怪他人,所以好到絕頂。姨婆曰:「你不須如此說,我知你委曲咯。」\\\\	住了幾日,珊瑚之母走來見女曰:「你母相隔得遠,一向唔知。今聞得女壻既寫分書我女,為何不囘母家,而在此攪擾姨婆,因乜緣故?」珊瑚曰:「女今無顏囘見父兄叔伯,就在此處繡花織布,粗茶淡飯,度日終身。」母曰:「女呀!睇你唔出做乜咁錯見?以你咁樣人材品貌,何憂冇好處。我要揀一個女壻,大多錢,好人品,又冇家婆拘束,然後嫁你。」珊瑚曰:「我聞忠臣不事二主,烈女不嫁二夫。女有一個家婆尚不能曉得奉事,更有何面目再入他家。母親如果要將女另嫁他人,女惟有投河吊頸,食藥自盡而已,斷不願偷生人世咯。」\\\\	詩曰:淡淡春風氣力微,池塘一水綠漪漪。蓮根自種深泥裏,不逐楊花到處飛。\\\\	話未完,喉頭哽咽,氣倒在地,哭不成聲。姨婆看見,眼中出淚,話其母曰:「你勿苦逼佢,由得佢咯。你逼佢太過,佢一時淺見,輕生個陣點算好呀!」其母亦拭淚而言曰:「唔知點樣解,天生得你個壞鬼女,有好處你唔行,有好人你唔做,其母心盲,未分好醜。重來掛念個的惡家婆,自怨唔奉事得佢透徹。你嫌佢羞磨得你少麽!制節得你少麽!提起個昏婆,我就想咬佢兩啖,你重唔捨得佢,係你賤咯!老母做主張尋訪好頭路,你去要有得食,有得着,你唔肯去,甘願捱饑抵餓,問你賤唔賤!你餓死,勿怨我老母;你冷死,勿怨我老母。你唔遵我講,我此後割斷条腸,總之作生少你一個。個吓唔慌重來望吓你!」珊瑚只管哭,其母只管罵,姨婆只管兩便開解。其母見女意終難轉,遂抽身抽勢,發脚就行。留他食飯,忿忿不答,出到門口,囘頭以手指珊瑚曰:「自後我唔認你做女,你亦不用認我做老母。」話完,忙忙而去。寫得老母火氣句句如生。其母去後,珊瑚遂在姨婆之處守志安居。\\\\	「忠孝節義」四字,為萬古綱常,頂天立地人物。此四個字,如大祠大廳之有四柱。祠廳之內,如簷前花板、板障花窗,可以粉飾浮誇,穿崩鬦湊;獨至四條大柱,須用堅石,須用實木,自頭到脚都要咁堅,都要咁實。外面雖然質樸,其中硬直不移,然後可以頂住棟樑,撐支大廈。天地之間須有忠孝節義等人,然後可以扶植綱常,轉移風俗。若使並無忠孝節義,個個俱是奸淫邪盜之人,吾恐日月無光,天翻地覆矣。忠孝節義,天上地下稱為四大名家。吾謂做忠臣難做,節婦更不易。少年之婦,曉得從一而終,立志不肯再嫁。無奈死者之骨肉未寒,而外家之親戚紛紛到門相勸,話有好頭路、好人家,早宜出脚。於是亞姑來勸者有之,亞姨來勸者有之,亞妗來勸者有之,而為之母者,更不知幾多甜言蜜語矣。媒人婆、竹筍䯻,又不知幾多花言巧語矣。若非鐵石心肝,未必不為其所動。今珊瑚之被逐出,夫雖未死,而恩情已斷矣。夫不以佢為妻,家婆不以佢為新婦矣,而猶情念故夫,心存孝道。老母幾番辱罵,百折不囘,節孝之心,可貫天日。吾願世之為婦道者,當繡其像,以香花奉之。\\\\	橫紋柴自珊瑚出門之後,招集做媒人等來吩咐曰:「我有好仔,唔憂冇新婦。你等媒婆,務宜代我尋一個好女子,送年庚入來。婚姻事成,我自有厚謝。別人謝媒婆,送銅錢二百,我謝媒婆,微微薄薄都要封銀兩大員。」各媒人領命而去,四處尋訪。誰知橫紋柴之名通傳遠近,各家父母見了佢個後枕就怕了九分,誰肯將女嫁佢個仔呢?是以兩年之久,都無一紙年庚入屋。橫紋柴嘆曰:「㗇,㗇!真正古怪唔通。我間屋唔好住?我的飯唔好食?為何總無人共我做親家呢?實在難明其故咯。」人人都明,總係自己唔明。\\\\	因見二成長大,不得不與他計策成婚。第二個新婦娶姓周,名呌臧姑。初歸入門,橫紋柴教之以孝順:「要低頭下氣,奉事家婆,千祈勿學我從前大新婦個的醜品。果然依你個句說話。你要好過佢為是。論起番來,你好我好。做家婆有乜唔愛新婦呢!總係做新婦唔明,家婆多的怒氣。有時家婆唔明,做新婦多的屈氣。你肯聽我教,我就心頭跌落脚踭筋咯。」\\\\	誰知二成個老婆名臧姑,其實呌做冇天裝,花號又呌做霸巷鷄乸。花號亦新。家婆話佢一句唔中意,佢就頂嘴十幾句。朝朝睡到日高三丈,然後起身。要治家婆洗碗洗碟,煮菜煮飯;家婆唔肯做,就大聲喝罵:「幾十歲人,各樣工夫唔做得的,唔通飯都唔煮得餐食吓?你估同我地後生,慢慢梳光頭,搽了粉,戴好花,又要扎周致个雙脚麽!」橫紋柴有時落得水多,落得水少,其飯煮得太軟太硬,臧姑就沉吟密咒,好似稟神咁樣稟,又罵老龜婆,又罵老狗乸。被橫紋柴聽知,怒曰:「你來咒我媽?」臧姑凸起眼睛曰:「我就咒你,你點樣惡法呀!我唔怕你惡,共你打清,然後食飯都做得。」話完即捲起衫袖,扎緊包頭帶,抽身抽勢,裝模作樣,好似猛虎下山想人肉食。原來臧姑生得又高又大,又肥又壯,又兇又惡。橫紋柴見其兇氣滿面,當時怕了三分。及至臧姑發起威來,橫紋柴即走出門外,大聲呌苦救命,圩咁嘈,蝦咁跳,話:「唔知乜頭路,娶着個的衰家狗,專門制治我!我一生純善,有鄰里所知,何嘗有你個的後生咁惡。豈有此理,新婦惡過家婆,你話難唔難呢!」臧姑聽聞,置之不理,皆掩口而笑。是晚家婆新婦企住門口,大鬧一場。橫紋柴咒至三更收功,臧姑偏咒至四更,然後收口。橫紋柴知自己鬦他不住,忍氣吞聲。\\\\	詩曰:臧姑偏要治家婆,只為家婆惡得多。嫩草怕霜霜怕日,惡人自有惡人磨。\\\\

% Quadratic Formula
\section*{Quadratic Formula}
\[
x = \frac{-b \pm \sqrt{b^2 - 4ac}}{2a}
\]

% Geometric Summation
\section*{Geometric Summation}
\[
S_n = a \frac{1 - r^n}{1 - r} \quad \text{for } r \neq 1
\]

% Definition of e
\section*{Definition of e}
\[
e = \lim_{n \to \infty} \left(1 + \frac{1}{n}\right)^n
\]

% Taylor Series for sin(x) and cos(x)
\section*{Taylor Series for sin(x) and cos(x)}
\[
\sin(x) = x - \frac{x^3}{3!} + \frac{x^5}{5!} - \frac{x^7}{7!} + \cdots
\]
\[
\cos(x) = 1 - \frac{x^2}{2!} + \frac{x^4}{4!} - \frac{x^6}{6!} + \cdots
\]

% Green's Theorem
\section*{Green's Theorem}
\[
\oint_C \left( P \, dx + Q \, dy \right) = \iint_D \left( \frac{\partial Q}{\partial x} - \frac{\partial P}{\partial y} \right) \, dA
\]

% Maxwell's Equations
\section*{Maxwell's Equations}
\[
\nabla \cdot \mathbf{E} = \frac{\rho}{\epsilon_0} \quad \text{(Gauss's law for electricity)}
\]
\[
\nabla \cdot \mathbf{B} = 0 \quad \text{(Gauss's law for magnetism)}
\]
\[
\nabla \times \mathbf{E} = -\frac{\partial \mathbf{B}}{\partial t} \quad \text{(Faraday's law of induction)}
\]
\[
\nabla \times \mathbf{B} = \mu_0 \mathbf{J} + \mu_0 \epsilon_0 \frac{\partial \mathbf{E}}{\partial t} \quad \text{(Ampère's law with Maxwell's correction)}
\]

% General Theory of Relativity
\section*{General Theory of Relativity}
\[
R_{\mu\nu} - \frac{1}{2} g_{\mu\nu} R + g_{\mu\nu} \Lambda = \frac{8 \pi G}{c^4} T_{\mu\nu}
\]

% Gödel's Incompleteness Theorem
\section*{Gödel's Incompleteness Theorem}

Any consistent formal system that is expressive enough to encode arithmetic contains true but unprovable statements.


Sed ut perspiciatis, unde omnis iste natus error sit voluptatem accusantium doloremque laudantium, totam rem aperiam eaque ipsa, quae ab illo inventore veritatis et quasi architecto beatae vitae dicta sunt, explicabo. Nemo enim ipsam voluptatem, quia voluptas sit, aspernatur aut odit aut fugit, sed quia consequuntur magni dolores eos, qui ratione voluptatem sequi nesciunt, neque porro quisquam est, qui dolorem ipsum, quia dolor sit amet consectetur adipisci[ng] velit, sed quia non numquam [do] eius modi tempora inci[di]dunt, ut labore et dolore magnam aliquam quaerat voluptatem. Ut enim ad minima veniam, quis nostrum[d] exercitationem ullam corporis suscipit laboriosam, nisi ut aliquid ex ea commodi consequatur? [D]Quis autem vel eum i[r]ure reprehenderit, qui in ea voluptate velit esse, quam nihil molestiae consequatur, vel illum, qui dolorem eum fugiat, quo voluptas nulla pariatur?


\begin{tableau}
  {                       % begin tree preamble
      line no sep= 2cm,   % distance of tree from line numbers
      for tree={s sep=10mm}, %control horizontal spread of branches
  }
  [P  
      [P\rightarrow Q
          [ \neg Q
              [\neg P, close]
              [Q, close]
          ]
      ]
  ]
  \end{tableau}
  
  \begin{tableau}
  {
      line no sep= 1.5cm,
      just sep= 1.5cm,  % Set separation of justification
  }
  [(P\wedge Q)\rightarrow R), just={Premise}
      [\neg(P\rightarrow (Q\rightarrow R)), just={Negated conclusion}
          [P, just={from (2)}
              [Q, just={from (2)}
                  [\neg R, s sep=30mm, just={From (4)} %Note "s sep" to spread fork below
                      [\neg(P\wedge Q),  just={Alternatives from (1)}
                          [\neg P, close, just={Alternatives from (7)}
                          ]
                          [\neg Q, close
                          ]
                      ]
                      [R, close]
                  ]
              ]
          ]
      ]
  ]
  \end{tableau}
  
  
  
  \begin{tableau}
      {line no sep=1.5 cm, 
      just sep=1.5cm,
      vertical/.style={
      before drawing tree={not ignore edge, edge=draw},
      close with=$\times$
      },
      }
  [((P\wedge Q)\vee R), just={Premise}
      [\neg\neg(\neg P\vee\neg R, just={Negated conclusion}
          [(\neg P\vee\neg R), just={From 2}
              [P\wedge Q, just={Alternatives from 1}
                  [P, just={from 4}
                      [Q, just={From 4}
                          [\neg P, close={5}]
                          [\neg R, just={Alternatives from (3)}
                              [\uparrow
                              ]
              ]]]]
              [R
              [,vertical
                  [,vertical
                      [\neg P, 
                          [\uparrow
                          ]
                      ] %and now we have two
                      [\neg R, close] %brances added
  ]]]]]]
  \end{tableau}
  
  \begin{tableau}
      {line no sep=1.5 cm, 
      just sep=1.5cm,
      vertical/.style={
      before drawing tree={not ignore edge, edge=draw},
      close with=$\times$
      },
      }
  [\neg(P\wedge Q), just={Premise}
      [Q\wedge R, just={Premise}
          [\neg\neg P, just={Premise}
              [\neg P, close={3,4}]
              [\neg Q, just={From 1, $\neg(\Phi \wedge \Psi)$}
                      [Q, just={From 2, $\Phi \wedge \Psi$}
                          [R, close={4,5}]
  ]]]]]]
  \end{tableau}
  
  \begin{fitch}
      \fj  A \\
      \fa \fh B \\
      \fa \fa A \\
      \fa  B \rightarrow A \\
  A \rightarrow (B \rightarrow A) \\
  \end{fitch}


  \chapter{《自序》}\chapter{《自序》}\chapter{《自序》}\chapter{《自序》}\chapter{《自序》}\chapter{《自序》}\chapter{《自序》}\chapter{《自序》}\chapter{《自序》}\chapter{《自序》}\chapter{《自序》}\chapter{《自序》}\chapter{《自序》}\chapter{《自序》}\chapter{《自序》}\chapter{《自序》}\chapter{《自序》}\chapter{《自序》}\chapter{《自序》}\chapter{《自序》}\chapter{《自序》}\chapter{《自序》}\chapter{《自序》}


\end{document}
