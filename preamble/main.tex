
% *-----------------------------------------------------------------------*
% | Fonts and typography                                                  |
% *-----------------------------------------------------------------------*

% Set CJK main font (for Chinese/Japanese/Korean characters)

% \setmainfont{Times New Roman}
\setCJKmainfont{BabelStone Han}
% doesn't work
% \setCJKmainfont{JyutcitziWithSourceHanSerifTCRegular}[
% Renderer=Basic,
% UprightFont = * ,
% FallbackFonts={BabelStone Han}
% ]



% You can also use \newfontfamily for custom non-CJK fonts if needed
% \setCJKmainfont{JyutcitziWithPMingLiURegular}[Path = ./, Extension = .ttf]
% \setCJKmainfont{JyutcitziWithSourceHanSerifTCRegular}[Path = ./, Extension = .ttf]



\newfontfamily{\jczPMingLiU}{JyutcitziWithPMingLiURegular}[Path = ../fonts/, Extension = .ttf]
% This has the best rendition for latin characters 
\newfontfamily{\jcz}{JyutcitziWithSourceHanSerifTCRegular}[Path = ../fonts/, Extension = .ttf]
\newfontfamily{\batang}{batang}[Path = ../fonts/, Extension = .ttf]
\newCJKfontfamily\koreanfont{Batang}[Path = ../fonts/, Extension = .ttf]




% *-----------------------------------------------------------------------*
% | Formatting     |
% *-----------------------------------------------------------------------*
% Set global paragraph indentation and spacing
\setlength{\parindent}{2em} % Adjust this value for the desired indentation
\setlength{\parskip}{0pt}   % No space between paragraphs

% for quotes
\usepackage{epigraph} 


\makeatletter
\renewcommand{\@makefntext}[1]{\jcz{\@thefnmark.} #1}
\makeatother
% to control itemise spacing
\usepackage{enumitem}


% Load ruby package for furigana (Ruby text)
\usepackage{ruby}
% \renewcommand{\ruby}[2]{%
%   \ruby{\jcz{#1}}{\jcz{#2}}%
% }


% *-----------------------------------------------------------------------*
% | Chinese and Soochow Numerals               |
% *-----------------------------------------------------------------------*

% Define Chinese numerals for numbers 1-99
% 〇〡〢 〣 〤 〥 〦 〧 〨 〩 十 〹 〺 卅
% Include the numerals file
\input{styles/numerals.tex}



% for vertical Chinese boxes
\usepackage{graphicx} % for \rotatebox

\newfontlanguage{Chinese}{CHN}

\setCJKfamilyfont{BabelStoneVert}[RawFeature={vertical;+vert},Script=CJK,Language=Chinese,Vertical=RotatedGlyphs]{BabelStone Han}

\newcommand*\CJKmovesymbol[1]{\raise.35em\hbox{#1}}
\newcommand*\CJKmove{\punctstyle{plain}% do not modify the spacing between punctuations
  \let\CJKsymbol\CJKmovesymbol
  \let\CJKpunctsymbol\CJKsymbol}

% Define a new environment for vertical text
\newcommand{\VertCell}[1]{\rotatebox{-90}{\CJKfamily{BabelStoneVert}\CJKmove #1}}


% IDC - ideographic description characters
% https://en.wikipedia.org/wiki/Chinese_character_description_languages#Ideographic_Description_Sequences

\usepackage{graphicx}

\newcommand{\superimpose}[2]{{%
  \ooalign{%
    \hfil$\m@th\text{#1}\@firstoftwo\text{#2}$\hfil\cr
    \hfil$\m@th\text{#1}\@secondoftwo\text{#2}$\hfil\cr
  }%
}}


% Define the \tb command

\newcommand{\tb}[2]{%
\scalebox{2}[1]{
\ooalign{%
    \hfil\raisebox{0.25em}{\text{\scalebox{0.33}{#1}}}\hfil\cr % Top text, squished and raised
    \hfil\raisebox{-0.25em}{\text{\scalebox{0.33}{#2}}}\hfil\cr % Bottom text, squished and lowered
  }%
  }
}

% The \lr command - can be combined with \tb
\newcommand{\lr}[2]{
  \scalebox{0.5}[1.0]{#1}\scalebox{0.5}[1.0]{#2}\!\!
}

% Define the \ul command for upper left positioning - for characters like 疒
\newcommand{\ul}[2]{%
  \ooalign{%
    \hfil#1\hfil\cr  % Top text (unscaled)
    \hfil\hspace{0.3em}\scalebox{0.8}{#2}\cr % Bottom text (scaled and raised)
    % \hfil\raisebox{0.2em}{\scalebox{0.5}{#2}}\hfil\cr % Bottom text (scaled and raised)
  }%
}

% Define the \tone command for upper right positioning of a diacritic
\newcommand{\tone}[2]{%
  \ifnum#2=1 \toneone{#1}%
  \else\ifnum#2=2 \tonetwo{#1}%
  \else\ifnum#2=3 \tonethree{#1}%
  \else\ifnum#2=4 \tonefour{#1}%
  \else\ifnum#2=5 \tonefive{#1}%
  \else\ifnum#2=6 \tonesix{#1}%
  \else\ifnum#2=7 \tonedouble{#1}%
  \else
    % Fallback: treat #2 as literal diacritic mark
    \ooalign{%
      \hfil#1\hfil\cr  % Main text (unscaled)
      \hfil\hspace{0.9em}\raisebox{0.3em}{\scalebox{0.8}{#2}}\hfil\cr % Tone mark (scaled and raised)
    }%
  \fi\fi\fi\fi\fi\fi\fi
}

% Define Cantonese tone commands (1-6)
\newcommand{\toneone}[1]{%
  \ooalign{%
    #1\cr
    \hspace{0.9em}\raisebox{0.85ex}{\rlap{\scalebox{0.6}{'}}}\cr
  }%
}
\newcommand{\tonetwo}[1]{%
  \ooalign{%
    #1\cr
    \hspace{0.7em}\raisebox{0.95ex}{\rlap{\scalebox{0.6}{ˊ}}}\cr
  }%
}
\newcommand{\tonethree}[1]{%
  \ooalign{%
    #1\cr
    \hspace{0.7em}\raisebox{0.95ex}{\rlap{\scalebox{0.6}{ˋ}}}\cr
  }%
}
\newcommand{\tonefour}[1]{%
  \ooalign{%
    #1\cr
    \hspace{0.9em}\raisebox{0.85ex}{\rlap{\scalebox{0.6}{"}}}\cr
  }%
}
\newcommand{\tonefive}[1]{%
  \ooalign{%
    #1\cr
    \hspace{0.7em}\raisebox{0.35ex}{\rlap{\scalebox{1}{\rotatebox{70}{゛}}}}\cr
  }%
}
\newcommand{\tonesix}[1]{%
  \ooalign{%
    #1\cr
    \hspace{0.5em}\raisebox{0.35ex}{\rlap{\scalebox{1}{゛}}}\cr
  }%
}


\usepackage{stackengine}

% for laddering
\usepackage{calc} % Needed for arithmetic in lengths
% *-----------------------------------------------------------------------*
% | Separators and formatting                                              |
% *-----------------------------------------------------------------------*

