\chapter{美感}

\section{美感絕對不是主觀的}

美感絕對不是主觀的,這是極度懶惰的相對主義。美觀是有邏輯的,否則不可能出現漢字圈一致認為殘體字醜死的現象。

你說推廣粵切字比推廣英語難。這是false的,嚴重直線邏輯,也完全機理錯誤。你根本完全沒有讀清楚我寫的東西。懂不懂英語根本不關事,問題在於拉丁字母的語文構建是逆民意和逆漢字語文美感的。粵切字反而是順民意和順漢字語文美感的,或至少沒有拉丁字母這麼嚴重。

你第三點也是完全不關事。他們傾慕英國文化,不意味一個漢字和拉丁字母混合的文章會雞犬升天,更不意味一個全拉丁字母的粵文會被他們學習。反而,用拉丁字母的粵文,所享受的尊重會更加低。是四不像。

粵切字設計上已經完成,短期上不會再改,因為暫時沒有任何的空間或需要改動。群眾開始使用後,理則出現新發展空間,就反而可以再調整。但重新設計是不會出現的。

所以呢,正字派係完全且永遠解決唔到文言分離,粵語語文構建不全,我手寫唔到我口嘅問題。po1根本就係「棵」嘅白讀音。但係呢度就有人唔小心得意得濟,「搵」(發明)咗個本字出黎。
本字考係非常好玩,發明過程令人陶醉—問題唔係在於發明,而係個發明過程不成系統,內無理則。文法仿效語法,故此語法必須係建基於有理則嘅文字上,繼而理則地呈現出黎,文法先至會相繼承語法,變得有理則。而當文法有理則,語文先可以有邏輯。

