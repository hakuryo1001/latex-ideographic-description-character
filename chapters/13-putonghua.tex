\chapter{普通話}

\section{講普通話係殖民嘅行為}
講普通話係殖民嘅行為。大義凜然仲要大條道理叫囂畀人歧視根本就係腦入水。唔係冇事就手足有事就蝗蟲,係發現原來我地信錯人,根本就唔係真係同我地企喺同一陣線,亦唔係真係同我地同心同德。佢只不過係透過香港黎凸顯自己係超級中國人。佢覺得有權喺香港講普通話,冇可能唔知個殖民性。姐係話,佢根本就只不過係兜個圈但之後都係殖民咁解。

唔好講到佢好\ul{疒}{}. 佢畀人窒唔畀講普通話,香港人日日如是,呢樹要同你講英文嗰樹同人講普通話,一黎個自價不菲嘅中國人就要即刻遷就轉台香港人都未嗲,邊度輪到佢嘈?仲有,死咁多人,佢呢啲雞毛蒜皮嘅野搞到滿城風雨,佢仲好意思?

我自己係非常非常想世界各地嘅人同我地行埋一齊,但係咪覺得你同我地行埋就好巴閉好大支野要乜都包容。粵語係香港嘅語言,而且應該係至尊語言。你入黎香港人無任歡迎,但唔好妹仔大過主人婆客家當地主。好煩。好討厭。


\section{中文嘅引經義 忱}

乜嘢係引經義忱?姐係「引經」嘅「義忱」。「引經」就係「引經據典」嘅「引經」,「義忱」就係「以其為合義嘅諗法,思緒,意識聽大,愛思糾結」。「引經義忱」就係中文世界裏面嗰種話寫文章講道理拗論點理應要引經據典先至好嘅主張。係士大夫同思哲自瀆愛好者最鍾意最󰹞󱥡嘅嘢。

引經義忱係一種可惡齷齪中世紀茅坑嘅八股混帳。喺漢字專用嘅世界裏面尤其。廣東話嘅文字題係轉咗做粵切字之後,引經義忱對引經者可以顯得高人一等嘅應該唔會削弱得去邊,但係人人都可以󰇞引經義忱嘅可能性就會大減,整到淨係得返啲真係有料嘅人先至可以引經,冇料扮四條嘅人就會因為門檻高左所以自行消失,亂咁引經博大霧造成嘅思哲污染就會自動散󱎦󰐢。

但係只要嗰日一直未到嚟,我地遇到任何以引經為樂嘅人,我地到絕對唔好。千祈唔好同佢摙過。因為絕對冇好處。唔玩呢個遊戲,只會俾人覺得你冇料,你廢,你玩唔掂先至喺度發爛渣。輸者無抗議之直。呢個現象一直都冇徹底打破: 抨擊姨得最犀利嘅,不外乎胡適嘅「八不主義」、 五四嘅新文化同新文學運動嗰陣嘅文風新倡議,同埋毛澤東主張嘅「大眾文學」。雖然股民同文言已經係所有一個需要用語文推進發展嘅範疇中被白話文所取代,但係呢種嘅引經義忱沒而不歿,陰魂不散,揮之不去,仲係死纏爛打。之所以係咁,有所以由官話演化出嚟嘅「白話文」再次有文言分離嘅跡象,漸漸演變成為「白言文」。


引經義忱最揦脷之處,就係佢驅使同鼓勵,趨逳啲引經言忱越嚟越難名,越嚟越難拆,唔搞到你𢱑晒頭都唔放棄,引嘅經典越刁鑽就越顯得你學識淵博,用嘅詞越難讀越睇唔明月冇辦法望文生義就越顯得你思考深邃。引經言忱,咁樣催生左一種秘語言忱:引經義忱秘語義忱。

秘語義忱天下就係一個用舊語主宰今事嘅義忱。喺呢個義忱之下人會變得思哲上不誠實,唔老實,爽韰為上,真想義理遺下。人只求得到嗰一下嘅韰,同埋自己身邊群組嘅認同:到底有冇道理,有冇玄理,有冇義理,話叉知佢。而正正因為咁樣,冇晒動力去以個人,獨立,新穎,批判性嘅視覺去剖析事情。冇新嘅言忱,冇新嘅意忱,冇——嘅義忱,一切都係舊酒新樽。賦予墨水靈魂嘅唔係真、實、啱、確,嘅玄理同玄義,而係死念。墨水都變得污穢。

懶醒,懶而不醒,就係一個清醒同誠實有勇氣凝視真理嘅人睇嗰啲滿口引經言忱嘅人嘅唯一結論。

\section{}
講普通話󱝚人,󰉥亾么文子力子央丌天,󰧲󱃲。󱪙󰧵講󱝚同󰻆仔我佢狗牌;󱪙街󰧵講普通話󱝚,話「󷋜」󱝚,󱪙好󱕡禮貌󰖱可𠄡可以講普通話󱝚,可以𠄡嬲,打佢𠄡告,仲要吿返佢轉頭,以佢侮辱國語,公眾地方講普通話所以公眾地方行為不檢,問人可唔可以講普通話所以告佢「要求他人講普通話」。所有嘅普通話老師都要罰錢。所有普通話嘅電影都要同時間禁止兼收稅,打孖上咁打壓。

我地要將講普通話變成罪行,變成冇禮貌嘅行為。我地要將所有人將講普通話睇到成好似隨街屙屎一樣咁鶻突嘅嘢做。

