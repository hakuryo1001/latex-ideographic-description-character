\chapter{其他腍審野}


\section{吳語小字對簡體字嘅立場}
如果你去睇下啲用吳語小字去構建吳語書面語嘅文嘅話,會見到佢哋基本上係一律否定同排斥中共簡體字嘅。但係,佢哋會喺某啲位選擇使用異體字,譬如用「躰」唔用「體」咁。呢個選擇好明顯係完全唔係因為字義原因,而係因為美感原因。佢哋想喺呢度郁.大必.々.嗰度郁.大必·々,以達到一種同漢字官白文有難以言喻嘅距離—操作同效果就好似日文中嘅漢字咁。我哋好大機會都要做啲類似嘅野。



\section{女書嘅故仔}
呢個故仔我自己第一次睇到嗰陣真係喊出黎。話說喺1982年武漢大學嘅一位宮哲兵博士生喺湖南一個叫永江嘅小鎮度發現到呢隻文字,仲發覺成個鎮只有女人識呢隻字,故此叫佢做「女書」。點解淨係得女人識呢?因為男尊女卑,女人冇得讀書,只可以靠撞靠估偷偷摸摸攞男人用嘅漢字,以其漢字讀音書寫自己嘅野,通常攞黎寫日記、繡喺扇同手巾仔上,記載嘅通常都係佢哋盲婚啞嫁、老公打老婆、姊妹情嘅哀詩。有人認為,女書可能有至少150年歷史,因為有啲天平天國嘅錢幣上有女書。而喺文革嗰陣,大量大量數之不盡有女書嘅用品全部畀人破四舊燒晒。

女書嘅自然傳人,即係細細個就寫女書大,有自然女書觸覺嘅人,喺2009年辭世。女書,已經冇自然傳人。


東亞有好多嘅文字演變都係咁,男人用漢字,新嘅、地位低下嘅、低劣嘅新文字,就由女人發明、使用、傳承。日本嘅假名,係好大程度上由啲貴族冇得參政嘅女室,為咗證明自己都可以搭嗲講到文學而自己演變出黎;韓國嘅諺文喺成宗發明咗之後冇幾耐就被一眾儒士所拋棄咒罵,但因為已經流落咗民間,女性就攞佢黎用,咁就傳承咗600年—而諺文正式上台,係要等到20世紀日本帝國將韓國嘅王帝同儒家全部殺晒,先至開始抬頭。

我成日都話,我地,係喺呢個漫長嘅六百年旅程嘅開端。係,係六百年。係好長,好痛苦。但係,係會成功嘅。係一定會成功。我地,就係喺呢個漫長嘅六百年旅程嘅開端。



\section{「方言」呢個狗牌}
當我地叫一樣我地認之為存在嘅「事物」為乜乜,賦予佢名稱時,我地從中係會奠定咗我哋對佢嘅認知,並確立咗我哋對佢存在嘅信念。而好多時候,呢個名字同佢係社會嘅用法,會衍生會維持某種政治或權利架構。呢個名字亦可能未必能夠「真實」「確切」反映到被描述物嘅現實特點。簡單啲黎講,就係我地叫某樣東西為X,我地就不由自主地認定咗X嘅存在,以及其存在模式。

咁講咁多同「方言」一詞有咩關係?如上所述,我地稱得一啲嘢為「方言」(同一啲嘢為「非方言」嘅物體),我地已經認定咗「方言」嘅存在,並且喺潛意識上奠定咗「方言」同「非方言」嘅標準。但係到底「方言」同「非方言」嘅界定線係啲乜嘢?係咪肆意無由(arbitrary)?而且就算呢個分類係人為嘅,係構建出黎嘅,佢有冇用?好唔好用?係可以擴大人類嘅知識,定係其實會導致人類概念模糊?用呢個詞黎剖釋世界,有冇咩政治同權力組態嘅衍生副作用?

先(試)處理一下「方言」呢個概念。我喺呢度我唔再以問題引渡,直接慳啲時間講小弟點睇算。

首先要講嘅「方言」一詞,無論喺大眾嘅觀點之中,抑或喺大中華學術界中,呢個詞語所表達嘅概念都同「dialect」有啲出入。雖然中西學術交流已經使到兩者逐漸趨同,但從其者在學術中所產生的蛛絲馬跡,我們仍然可以見到兩者有異。

「方言」同dialect嘅共通點就係佢地都係描述緊某種同「語言」相對嘅語碼(code)現象。呢個係釐定「方言」同「語言」嘅出發點,亦係普羅大眾對呢兩個詞嘅普通理解。

但係好明顯呢個理解係冇辦法自立,根本就自相矛盾,而且只要一直堅持「方言」同「語言」係相斥嘅關係,就必定無辦法成為一個內部邏輯通順嘅語言詮釋範式(interpretative paradigm). 原因好簡單,因為「語言」language 一詞,普遍概念上包括咗「方言」dialect嘅概念。你可以諗諗,方言如果唔係語言嘅一種存在形式,咁佢係啲乜野?如果「語言」嘅定義係「一個以人類口頭發聲按著某種邏輯規律以傳達信息資訊嘅系統」,咁「方言」又豈能不是一種語言嘅一種存在形式?

之所以會出現以上嘅類悖論情況,係因為我地冇釐清呢個同「方言」相對嘅「語言」嘅概念係啲乜東東。事實上,茲「語言」不同彼「語言」。

「方言」始終喺我哋日常嘅理解當中係相對於「語言」。我地好多時候講X係「語言」而唔係「方言」,言下之意可能係指X按某種標準而言比較「正式」;反之,當我地話X係「方言」而唔係「語言」,言下之意就係X「(唔夠)正式」。

用以上對「語言」同「方言」嘅理解,如果講得抽象同哲學一啲,就係當咗「語言」同「方言」係「一位謂詞」(1 place predicate).

有時候我地又會以以下嘅方式去演繹。我地可能會話X係Y嘅方言,而喺呢種講法係Y係「語言」,X係「方言」,而「語言—方言」係一種階級關係,Y(語言)支配住X (方言)。例子包括今日講到滿城風雨嘅教育局偉論:「粵語係漢語方言」。政治上冇咁具爭議嘅例子就可能有:African American English 係英文嘅方言/ Canadian French 係法文方言 / 北京話係官話方言 / 四邑話係粵語嘅方言 / Bavarian 係德語嘅方言。

但仲未完。有時候我地又會以以下嘅方式去演繹。我地可能會話X係C嘅方言,而Y係C嘅語言。例子:Spanish 係西班牙語言而 Catalan 係西班牙方言 / 普通話係中國語言而粵語係中國方言。

以上兩者都係將「方言」當為係「兩位謂詞」(2 place predicate):前者係「語言—方言」嘅兩位謂詞 Rxy where x 係方言 y 係語言;後者係「語言/方言—地方」嘅兩位謂詞 Rxc where x 係某種語碼,c係地方,而x係方言定係語言就視乎x係乜同c係乜。

到呢度「方言」呢個概念有咩問題應該已經可以略見端倪。好明顯,以上三個對「方言」嘅詮釋,係冇可能同時並立嘅。(呵呵我相信用以上嘅定義可以用數學證明出黎,但而家就唔搞呢範野啦)。

第二個問題就係以上三個嘅詮釋,都係只可以釐清「語言」同「方言」嘅關係,但係冇畀任何指示我地去決定咩時候咩係語言咩係方言同咩係咩嘅方言。簡單而言就係我地仲係冇釐定語言同方言嘅實質標準。

姑勿論以上三個詮釋明顯會要求有三個不同嘅標準呢個問題,就算我地暫且只專注求祈一個,我地都會發覺,個標準好難定。點解?因為有好多例外,同好難得到「普遍性」同「泛可用性」。呢個問題其實唔係好複雜嘅姐,但係要講就真係好煩水蛇春咁長,一句既之曰其理則為「語言只不過係有軍隊嘅方言咁解」。而正因茲原因,語言學家普遍都唔會嚴格定義「方言」係啲乜東東,只會當「方言」一詞為rule of thumb 速語,唔會胡亂定性乜乜語言為「方言」或「語言」 。

好啦,到戲玉啦。到底點解「方言」一詞有問題?

請循其本—我地開頭就講咗,我地社會用語中嘅詞彙同用法,係可以塑造我地對現實嘅理解。標籤某一種語碼為「方言」,係可以(亦幾乎必定會)產生龐大嘅政治作用,而呢個作用往往係具壓迫性,打壓性,同賤貶性。使用「方言」一詞黎稱呼同標籤某種語碼,就係會使被標籤者馬上蒙受語言權威同威望(prestige)嘅損失,而往往他者嘅損失,就係某者嘅(政治、經濟、文化)得益。例子實在太多,中國內對粵、吳、客、閩諸華語、法國對 Occitan, basque, 西班牙對 Catalan, 等等。

以上嘅原因適用於dialect同埋「方言」,但「方言」一詞就更加有問題。「方言」一詞除了包括晒dialect嘅絕大部分潛意思之外,佢係仲額外承載住濃郁但難以言喻嘅大中華思想同中華中心主義。呢個好複雜,難以詳述,但小弟盡下力。第一,「方言」一詞歷史上係相對於「雅言」,而雅言就係天朝上國嘅語言(口語)。所以按呢個邏輯,家下天朝上國嘅雅言係普通話,所以粵語、吳語、客家語等等就通通都係「方言」。呢個邏輯體系以前仲比較強,而家就已經弱,但係始終係普羅華人大眾中仍然有無色無聲嘅影響。要知道,喺清末民初搞翻譯書院嗰陣,藏語、蒙古語、粵語、甚至英語、法語、日語,都列為「方言」。梁啟超當初讀西方邏輯嗰陣,唔知係乜,都挾硬將「邏輯」一學列為「方言」。

二、西方嘅dialect同(standard) language嘅普羅釐定界線,唔係「相互可通性」mutual intelligibility, 就係國界。中國嘅「方言」除了一向兩者,仲包括文字。如果個語言用得漢字,就係「方言」。所以按呢個邏輯,即使北京人去上海去香港去潮州去台南聽唔明上海話廣東話潮州話閩南話,佢地都係會話佢哋冚把爛係方言。在極端啲嘅連日文,甚至歷史上用過漢字嘅韓語同越南語都唔放過話係方言。

以上兩點加埋「方言」一詞所可以帶來嘅政治作用,應該足夠說明點解「方言」茲詞理應慎用。



