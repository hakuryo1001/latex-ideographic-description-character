% ==============================
% JAPANESE KATAKANA STACKING MACROS
% ==============================

% (1) Tight top-bottom stack for kana-like fusions
\newcommand{\jtb}[2]{%
  \scalebox{2}[1]{%  % ← elongate horizontally by 2×
    \ooalign{%
      \hfil\raisebox{0.45em}{\scalebox{0.65}{#1}}\hfil\cr % top element
      \hfil\raisebox{-0.1em}{\scalebox{0.65}{#2}}\hfil\cr % bottom element
    }%
  }%
}


% (2) Tight left-right fusion (for kana appended to the side of kanji)
\newcommand{\jlr}[2]{%
  \ooalign{%
    \hfil#1\hfil\cr
    \hspace{0.55em}\raisebox{0.05em}{\scalebox{0.45}{#2}}\hfil\cr
  }%
}

% (3) Inner overlay — kana or Latin letter inside the kanji body (like 話ョ or 鯨ク)
\newcommand{\jin}[2]{%
  \ooalign{%
    \hfil#1\hfil\cr
    \hfil\raisebox{0.05em}{\scalebox{0.45}{#2}}\hfil\cr
  }%
}

% (4) Top-right diacritic or kana addition (for kana marks like small ョ, or dakuten)
\newcommand{\jur}[2]{%
  \ooalign{%
    #1\cr
    \hspace{0.75em}\raisebox{0.75em}{\scalebox{0.45}{#2}}\cr
  }%
}

% (5) Top-left variant (for kana on upper-left of kanji)
\newcommand{\jul}[2]{%
  \ooalign{%
    #1\cr
    \hspace{-0.75em}\raisebox{0.75em}{\scalebox{0.45}{#2}}\cr
  }%
}
