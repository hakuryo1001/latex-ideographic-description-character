\documentclass[a5paper, 12pt, openany]{book} % A5 paper size

% \usepackage[paperwidth=148mm, paperheight=210mm, top=1.27cm, bottom=1.27cm, inner=1.9cm, outer=1.27cm, headsep=1.5cm, footskip=1.75cm]{geometry} % Custom dimensions and margins
% Adjusted margins to ensure space for page numbers
\usepackage[paperwidth=148mm, paperheight=210mm, 
         top=1.27cm, bottom=2.5cm, inner=2.2cm, outer=1.27cm, 
         headsep=1.5cm, footskip=1.5cm]{geometry}


\usepackage[utf8]{inputenc}
\usepackage{ctex}

\usepackage{fancyhdr}
\pagestyle{fancy}
\fancyhf{} % Clear all header and footer fields
% \fancyfoot[C]{\thepage} % Center the page number at the bottom
% Define left and right page numbering
\fancyfoot[LE]{\thepage} % Left side for even pages
\fancyfoot[RO]{\thepage} % Right side for odd pages
% Ensure the chapter pages (plain style) also have this layout
\makeatletter
\let\ps@plain\ps@fancy
\makeatother

% *-----------------------------------------------------------------------*
% | Fonts and typography                                                  |
% *-----------------------------------------------------------------------*

% Set CJK main font (for Chinese/Japanese/Korean characters)

% \setmainfont{Times New Roman}
\setCJKmainfont{BabelStone Han}
% doesn't work
% \setCJKmainfont{JyutcitziWithSourceHanSerifTCRegular}[
% Renderer=Basic,
% UprightFont = * ,
% FallbackFonts={BabelStone Han}
% ]



% You can also use \newfontfamily for custom non-CJK fonts if needed
% \setCJKmainfont{JyutcitziWithPMingLiURegular}[Path = ./, Extension = .ttf]
% \setCJKmainfont{JyutcitziWithSourceHanSerifTCRegular}[Path = ./, Extension = .ttf]



\newfontfamily{\jczPMingLiU}{JyutcitziWithPMingLiURegular}[Path = ./fonts/, Extension = .ttf]
% This has the best rendition for latin characters 
\newfontfamily{\jcz}{JyutcitziWithSourceHanSerifTCRegular}[Path = ./fonts/, Extension = .ttf]
\newfontfamily{\batang}{batang}[Path = ./fonts/, Extension = .ttf]
\newCJKfontfamily\koreanfont{Batang}[Path = ./fonts/, Extension = .ttf]




% *-----------------------------------------------------------------------*
% | Quotes and formatting     |
% *-----------------------------------------------------------------------*

\usepackage{epigraph} 

\renewcommand{\contentsname}{目錄} % Traditional Chinese characters for "Contents"
% Set global paragraph indentation and spacing
\setlength{\parindent}{2em} % Adjust this value for the desired indentation
\setlength{\parskip}{0pt}   % No space between paragraphs

\setcounter{secnumdepth}{0} % no numbering for sections 
% Increase chapter title size in TOC
\usepackage{tocloft} % For customizing table of contents
\renewcommand{\cftchapfont}{\Large\bfseries} % Large and bold chapter titles
\renewcommand{\cftchappagefont}{\Large\bfseries} % Large and bold page numbers

\renewcommand{\figurename}{圗}

\makeatletter
\renewcommand{\@makefntext}[1]{\jcz{\@thefnmark.} #1}
\makeatother
% to control itemise spacing
\usepackage{enumitem}
% create index - run \makeindex in the document
\usepackage{makeidx}
\makeindex



% *-----------------------------------------------------------------------*
% | Math & Equations     |
% *-----------------------------------------------------------------------*
\usepackage{amsmath} % For advanced math formatting
\usepackage{amssymb} % For mathematical symbols
% \usepackage{tikz} % For drawing logic decision trees



% *-----------------------------------------------------------------------*
% | Table Management                                                      |
% *-----------------------------------------------------------------------*


\usepackage{graphicx}
\usepackage{array}
\usepackage{tabularx}
\usepackage{tabularray}

\usepackage{float}      % Add the float package


\usepackage[table,xcdraw]{xcolor}
% 

% Load ruby package for furigana (Ruby text)
\usepackage{ruby}
% \renewcommand{\ruby}[2]{%
%   \ruby{\jcz{#1}}{\jcz{#2}}%
% }


% *-----------------------------------------------------------------------*
% | Chinese and Soochow Numerals for chapter management                   |
% *-----------------------------------------------------------------------*

% Define Chinese numerals for numbers 1-99
% 〇〡〢 〣 〤 〥 〦 〧 〨 〩 十 〹 〺 卅

\newcommand{\soochowNumeral}[1]{%
  \ifnum#1<10
    \ifcase#1 〇\or 〡\or 〢\or 〣\or 〤\or 〥\or 〦\or 〧\or 〨\or 〩\fi%
  \else
    \ifnum#1<20
      〸\soochowUnits{\numexpr#1-10\relax}%
    \else
      \ifnum#1<30
        〹\soochowUnits{\numexpr#1-20\relax}%
      \else
        \ifnum#1<40
          〺\soochowUnits{\numexpr#1-30\relax}%
        \else
          \ifnum#1<50
            卅\soochowUnits{\numexpr#1-40\relax}%
          \else
            \ifnum#1<60
              〥十\soochowUnits{\numexpr#1-50\relax}%
            \else
              \ifnum#1<70
                〦十\soochowUnits{\numexpr#1-60\relax}%
              \else
                \ifnum#1<80
                  〧十\soochowUnits{\numexpr#1-70\relax}%
                \else
                  \ifnum#1<90
                    〨十\soochowUnits{\numexpr#1-80\relax}%
                  \else
                    \ifnum#1<100
                      〩十\soochowUnits{\numexpr#1-90\relax}%
                    \fi
                  \fi
                \fi
              \fi
            \fi
          \fi
        \fi
      \fi
    \fi
  \fi
}

% Helper command for units (ones digit)
\newcommand{\soochowUnits}[1]{%
  \ifnum#1=0
  \else
    \ifnum#1<4
      \ifcase#1 \or 一\or 二\or 三\fi%
    \else
      \soochowNumeral{#1} % Use Suzhou numeral for numbers greater than 3
    \fi
  \fi
}

\newcommand{\chinesenumeral}[1]{%
  \ifnum#1<10
    \ifcase#1 〇\or 一\or 二\or 三\or 四\or 五\or 六\or 七\or 八\or 九\fi%
  \else
    \ifnum#1<20
      十\chinesenumeral{\numexpr#1-10\relax}%
    \else
      \ifnum#1<30
        二十\chinesenumeral{\numexpr#1-20\relax}%
      \else
        \ifnum#1<40
          三十\chinesenumeral{\numexpr#1-30\relax}%
        \else
          \ifnum#1<50
            四十\chinesenumeral{\numexpr#1-40\relax}%
          \else
            \ifnum#1<60
              五十\chinesenumeral{\numexpr#1-50\relax}%
            \else
              \ifnum#1<70
                六十\chinesenumeral{\numexpr#1-60\relax}%
              \else
                \ifnum#1<80
                  七十\chinesenumeral{\numexpr#1-70\relax}%
                \else
                  \ifnum#1<90
                    八十\chinesenumeral{\numexpr#1-80\relax}%
                  \else
                    \ifnum#1<100
                      九十\chinesenumeral{\numexpr#1-90\relax}%
                    \fi
                  \fi
                \fi
              \fi
            \fi
          \fi
        \fi
      \fi
    \fi
  \fi
}

% Custom chapter title formatting with Chinese numeral chapter numbers
\usepackage{titlesec}


% Custom chapter title formatting with Chinese numeral chapter numbers
\titleformat{\chapter}[block] % 'block' means the title appears on a new line
  {\Huge\bfseries} % Font size and bold formatting for the title
  {\soochowNumeral{\thechapter}} % Chinese character for chapter number
  {1em} % Space between the number and the title
  {\Huge} % Custom style for the chapter title itself (can modify)

% Remove the default LaTeX behavior of forcing new chapters to start on a new page
\makeatletter
\renewcommand\chapter{\if@openright\cleardoublepage\else\clearpage\fi
  \thispagestyle{plain}%
  \global\@topnum\z@
  \@afterindentfalse
  \secdef\@chapter\@schapter}
\makeatother




% *-----------------------------------------------------------------------*
\begin{document}
% to avoid overfull hbox
\sloppy
% % \jcz{} must be run so the document can process jyutcitzi 
\jcz{} 





% % \jczSourceHan{}

\tableofcontents



\chapter{粵切字 Sample |  }

聲母

\begin{table}[H]
  \centering
  \begin{tabular}{|>{\centering\arraybackslash}m{2cm}|>{\centering\arraybackslash}m{2cm}|>{\centering\arraybackslash}m{2cm}|>{\centering\arraybackslash}m{2cm}|} 
    \hline
    \begin{tabular}[c]{@{}c@{}}b 比\\ ⿱\end{tabular} & \begin{tabular}[c]{@{}c@{}}p 并\\ ⿰\end{tabular} & \begin{tabular}[c]{@{}c@{}}m 文\\ ⿱\end{tabular} & \begin{tabular}[c]{@{}c@{}}f 夫\\ ⿰\end{tabular}  \\ 
    \hline
    \begin{tabular}[c]{@{}c@{}}d 大\\ ⿱\end{tabular} & \begin{tabular}[c]{@{}c@{}}t 天\\ ⿱\end{tabular} & \begin{tabular}[c]{@{}c@{}}n 乃\\ ⿰\end{tabular} & \begin{tabular}[c]{@{}c@{}}l 力\\ ⿰\end{tabular}  \\ 
    \hline
    \begin{tabular}[c]{@{}c@{}}z 止\\ ⿰\end{tabular} & \begin{tabular}[c]{@{}c@{}}c 此\\ ⿱\end{tabular} & \begin{tabular}[c]{@{}c@{}}s 厶\\ ⿱\end{tabular} & \begin{tabular}[c]{@{}c@{}}j 央\\ ⿱\end{tabular}  \\ 
    \hline
    \begin{tabular}[c]{@{}c@{}}g 丩\\ ⿰\end{tabular} & \begin{tabular}[c]{@{}c@{}}k 臼\\ ⿱\end{tabular} & \begin{tabular}[c]{@{}c@{}}h 亾\\ ⿰\end{tabular} & \begin{tabular}[c]{@{}c@{}}ng \scalebox{0.5}[1.0]{乂}\scalebox{0.5}[1.0]{乂}\\ ⿱\end{tabular} \\ 
    \hline
    \begin{tabular}[c]{@{}c@{}}gw 古\\ ⿰\end{tabular} & \begin{tabular}[c]{@{}c@{}}kw 夸\\ ⿰\end{tabular} & \begin{tabular}[c]{@{}c@{}}w 禾\\ ⿱\end{tabular} & \begin{tabular}[c]{@{}c@{}}m/ng 𫝀\\ \ \end{tabular}  \\ 
    \hline
  \end{tabular}
\end{table}
% 韻母

韻母

\begin{table}[H]
  \centering
  \resizebox{\textwidth}{!}{ % Adjust the width to fit within the page
    \begin{tblr}{
      colspec={|X[c]|X[c]|X[c]|X[c]|X[c]|X[c]|X[c]|X[c]|X[c]|X[c]|}, % Column alignment
      hlines, % Horizontal lines
      vlines  % Vertical lines
    }
        & \empty   & -i  & -u  & -m  & -n  & -ng  & -p  & -t  & -k \\
    /aa/ & aa \linebreak 乍  & aai \linebreak 介 & aau \linebreak 丂 & aam \linebreak 彡 & aan \linebreak 万 & aang \linebreak 生 & aap \linebreak 甲 & aat \linebreak 压 & aak \linebreak 百 \\
    /a/  &      & ai \linebreak 兮 & au \linebreak 久 & am \linebreak 今 & an \linebreak 云 & ang \linebreak 亙 & ap \linebreak 十 & at \linebreak 乜 & ak \linebreak 仄 \\
    /e/  & e \linebreak 旡 & ei \linebreak 丌 & eu \linebreak 了 & em \linebreak 壬 & en \linebreak 円 & eng \linebreak 正 & ep \linebreak 夾 & et \linebreak 叐 & ek \linebreak 尺 \\
    /i/  & i \linebreak 子 &      & iu \linebreak 么 & im \linebreak 欠 & in \linebreak 千 & ing \linebreak 丁 & ip \linebreak 頁 & it \linebreak 必 & ik \linebreak 夕 \\
    /o/  & o \linebreak 个 & oi \linebreak 丐 & ou \linebreak 冇 &      & on \linebreak 干 & ong \linebreak 王 &      & ot \linebreak 匃 & ok \linebreak 乇 \\
    /u/  & u \linebreak 乎 & ui \linebreak 会 &      &      & un \linebreak 本 & ung \linebreak 工 &      & ut \linebreak 末 & uk \linebreak 玉 \\
    /oe/ & oe \linebreak 居 &      &      &      &      & oeng \linebreak 丈 &      &      & oek \linebreak 勺 \\
    /eo/ &      & eoi \linebreak 句 &      &      & eon \linebreak 卂 &      &      & eot \linebreak 𥘅$_{\text{朮}}$ &      \\
    /yu/ & yu \linebreak 仒 &      &      &      & yun \linebreak 元 &      &      & yut \linebreak 乙 &      \\
    \end{tblr}
  }
  \caption{韻母}
\end{table}

聲調

\begin{table}[H]
  \jcz{}
  \centering
    \begin{tblr}{
      colspec={|X[c]|X[c]|X[c]|X[c]|X[c]|X[c]|},  % Equal-width columns and centered text
      hlines,  % Draw horizontal lines
      vlines   % Draw vertical lines
    }
      1 & 2 & 3 & 4 & 5 & 6 \\ 
      󰘠、󰘦 & 󰘡 & 󰘢 & 󰘣、󰘧 & 󰘤 & 󰘥 \\ 
      󰝰、󰝶 & 󰝱 & 󰝲 & 󰝳、󰝷 & 󰝴 & 󰝵 \\
      分 & 粉 & 訓 & 墳 & 憤 & 份 \\
    \end{tblr}
  \caption{切字 聲調}
\end{table}

\begin{table}[htbp]
  \jcz{}
  \centering
  \renewcommand{\arraystretch}{1.5} % Adjust row height
  \setlength{\tabcolsep}{4pt} % Adjust column padding
  \resizebox{\textwidth}{!}{
  \begin{tabularx}{\textwidth}{|X|X|X|X|}
  \hline
  % \rowcolor[HTML]{D0D0D0} 
  \textbf{坊間漢羅混用} & \textbf{漢字已整理版本} & \textbf{漢字粵切字混用(未組裝)} & \textbf{漢字粵切字混用(已組裝)} \\
  \hline
  咁都係果D嘢嘎啦,廿鯪蚊個餐又湯又剩唔通有得你食天九翅咩?求求其其有D肉有D菜蛋白質澱粉質撈撈埋埋打個白汁茄汁黑椒汁咁撐得你懵口懵面咪Lui返去返工返學返廠返寫字樓囉。唔係你估真係搵餐晏仔咁簡單啊。咁跟飯定跟意粉啊? 
  & 咁都係果啲嘢㗎啦,廿鯪蚊個餐又湯又剩唔通有得你食天九翅咩?求求其其有啲肉有啲菜蛋白質澱粉質撈撈埋埋打個白汁茄汁黑椒汁咁撐得你懵口懵面咪纍返去返工返學返廠返寫字樓囉。唔係你估真係搵餐晏仔咁簡單啊。咁跟飯定跟意粉啊? 
  & 丩今´都係丩个´大子¯野丩乍`力乍`,廿力正⁼蚊個餐又湯又剩𠄡通有得你食天九翅文旡¯?求々其々有大子¯肉有大子¯菜蛋白質澱粉質撈々埋々打個白汁茄汁黑椒汁丩今´止生゙得你懵口懵面文兮`力句¯返去返工返學返廠返寫字樓力个¯。𠄡係你估真係搵餐晏仔丩今`簡單⺍乍⁼。丩今´跟飯定跟意粉⺍乍`?
  & 󱜩都係󱟡󰦠野󱛒󰿒,廿󰻃蚊個餐又湯又剩𠄡通有得你食天九翅󰗘?求々其々有󰦠肉有󰦠菜蛋白質澱粉質撈々埋々打個白汁茄汁黑椒汁󱜩󰿽得你懵口懵面󰖚󰾠返去返工返學返廠返寫字樓󰼠。𠄡係你估真係搵餐晏仔󱜪簡單󰀓。󱜩跟飯定跟意粉󰀒? \\
  \hline
  \end{tabularx}
  }
\end{table}
  


\chapter{\ruby{囻}{󱼒}之語音}


今日,十月九號,係諺文日。

諺文日,又稱之爲「韓字日」({\koreanfont 한글날}),係南韓爲咗記念喺1446年,世宗大王公佈《訓民正音》({\koreanfont 훈먼정음})而定嘅國定假期。

北韓都有自己對應假期,定喺一月十五號。

之所以要咁大陣象舉國慶祝,係因為「諺文」嘅發明,解決咗朝鮮呢個國家嘅文字問題,事關喺諺文之前,韓國係冇自己嘅文字。諺文嘅發明,唔單止為韓語嘅「有音無字」「言文分離」提供咗解決方案,奠立咗「韓文」嘅基礎,仲為韓國奠基咗佢哋自己嘅獨立文化嘅基礎。如果倉頡係華夏文明嘅開端,話諺文係朝鮮文明嘅開端都唔係冇得拗。

\subsection*{諺文嘅發明背景}

「諺」嘅原意係「俗語」,顧名思義「諺文」一詞就係「表記俗語(朝鮮語)之文字」嘅意思。呢度其實都已經見到諺文嘅發明開端。

嗰陣時,韓語只不過係平民百姓嘅口語,貴族同士大夫雖然口講韓語,但係寫嘅就係漢字。想寫野,就必須用漢字。用漢字,你可以寫文言文。如果想將篇文言文用韓語讀返出黎,就要用一啲非常複雜、無咩系統嘅規矩,去將文言文句子轉化成為符合韓語語法嘅句子,先至得。如果你想寫韓語,姐係想「我手寫我口」,你淨係有漢字可以用。你就只可以用漢字,攞住漢字嘅音去寫韓語,亦即係通篇用「假借」嘅手法去寫,好聽你就係《萬葉集》,難聽啲你真係同「港女文」冇乜分別。而呢啲嘅查實係見招拆招嘅書寫手法,就衍生咗所謂嘅「鄉札」、「口訣」、「吏讀」。漢字本身要學就已經成本高,噉樣嘅文字秩序就令韓文嘅讀寫變得更加係難上加難。

朝鮮國嘅第四代國王世宗大王,好想喺佢嘅國家推行儒家禮教,但係佢寫用黎教育大眾嘅書,淨係可以用漢字,庶民根本睇唔明。佢嘅臣民,有野想講,都冇辦法喺書面上同佢表達。為咗解決呢個極度嚴重嘅「言文分離」問題,佢就下令要發明一隻新文字。喺《世宗原詔》度,佢話(原文係漢文):
\begin{quotation}
	國之語音,異乎中國,與文字不相流通,故愚民有所欲言,而終不得伸其情者多矣。予爲此憫然,新制二十八字,欲使人人易習便於日用耳。
	
\end{quotation}
用粵語白話文講嘅話就係:

\begin{quotation}
	我哋國家嘅語音,同中國嘅唔同,所以同中國嘅文字唔啱牙。正因為係咁,愚民(唔識字嘅平民)就算有野想講(有冤情),最終好多都冇辦法同國家申訴。我為呢個問題深感悲哀,對佢哋面對嘅情況深感憐憫。所以,我而家就整咗二十八個新字出黎,希望人人都可以好容易學識,畀喺佢哋日常生活度攞黎用。
\end{quotation}

世宗大王發明嘅諺文,都真係相當之驚為天人。呢隻文字,表音奇準,使用規則奇簡,字符同字符之間喺字形設計上有耐人尋味嘅隱性邏輯,仲將中國「天地人」同陰陽學說暗合咗入去,仲借鑒咗漢字嘅六書原則並加以闡發托展——基本上前無古人,後無來者。同樣係漢字衍生出黎姊妹書寫系統,比如西夏文、女真文、喃字、方塊壯字、日本假名等等,都冇諺文咁科學——諺文稱之為東亞文字最為犀利者實在當之無愧。

呢隻文字,真係「智者不終朝而會,愚者可夾旬而學」。意思姐係「聰明人唔使一個朝頭早就可以學識,就算係白癡都可以十日內搞掂」。

\subsection*{反對諺文同戀慕中華嘅士大夫}
世宗嘅諺文推出咗冇幾耐,就有人出黎大大聲聲反對。最有代表性嘅,就係一個叫崔萬理嘅儒家士大夫。佢寫咗一篇題為《反對創建韓文》(又名《上疏反對世宗推行諺文》)畀世宗睇嘅上疏。裏面開宗明義就話:
\begin{quotation}
	我朝自祖宗以來,至誠事大,一遵華制,今當同文同軌之時,創作諺文,有駭觀聽。儻曰諺文皆本古字,非新字也,則字形雖倣古之篆文,用音合字,盡反於古,實無所據。若流中國,或有非議之者,豈不有愧於事大慕華?	
\end{quotation}

我哋唔使翻譯晒出來都睇到佢嘅反對重點,就係「採用諺文,就係脫離中國,咁樣係違反『一遵華制』、『至誠事大』、『事大慕華』嘅國策。」乜嘢係「事大」呢?「事大」,姐係「事大主義」。「事大」一詞來源於《孟子》嘅「以小事大」。。「事大」裏面嘅「事」,可以理解為「服侍」。「事大」就係「為大行事」或者「服侍個『大』嘅野」。咁乜野(或者邊個)係「大」呢?就係中華,亦姐係中原皇朝,亦姐係中國。而「事大主義」注意,就係外交政策上視中原皇朝為中華(故此為「大」),以自己為「小」,而呢個外交政策上嘅體現,其中一部分就係文化上靠攏中國,將自己變成為「小中華」。崔萬理嘅意思就係,如果推諺文,就係違反朝鮮作為小中華嘅身分——言下之意就係,做得中華,就一定要用漢字。

佢之後又話:

\begin{quotation}
	自古九州之內,風土雖異,未有因方言而別爲文字者,唯蒙古、西夏、女眞、日本、西蕃之類,各有其字,是皆夷狄事耳,無足道者。《傳》曰:「用夏變夷,未聞變於夷者也。」歷代中國皆以我國有箕子遺風,文物禮樂,比擬中華。今別作諺文,捨中國而自同於夷狄,是所謂棄蘇合之香,而取螗螂之丸也,豈非文明之大累哉?	
\end{quotation}

簡單黎講,就係話自古以來,就算方言唔同,都唔會改用其他文字(換句話就係佢認定韓語只不過係方言)——而唔用漢字嘅,就係自作夷狄。由華夏文明去開發夷狄,令佢哋變成為中華,就係古而有之嘅大道理姐;調返轉頭本來已經喺華夏文明裏面,跳返出黎做野蠻人,邊有咁嘅道理架?之後佢又話朝鮮好耐之前(春秋戰國時期)就已經有中原嘅「箕子」蒞臨,朝鮮仲繼承咗佢所帶黎華夏文化添
——我哋嘅文化,根本就可以同中華有得揮。家下你另起爐灶發明諺文,係放棄中國而將自己化為冇文化嘅野蠻人,係放棄蘇合呢種香草嘅香,而換黎曱甴卵嘅行為,噉樣仲唔係文明大倒退?

\subsection*{諺文之後嘅發展}

崔萬理$_{\text{fi li fe le}}$$_{\text{bi li baa laa}}$嘅理性批鬥,都唔係好阻止到世宗大王推行諺文。諺文好快就喺民間度流出去散播。但係唔係個個都接受同肯採用。開初大部分嘅名門望族同士大夫多對諺文嗤之以鼻,繼續用佢哋嘅漢字,要到二十世紀上流社會先至出現啲毫無系統嘅漢諺混寫。諺文就反而喺低下階層,尤其是係女人同奴婢之間度廣泛採用。唔少本來唔識字,亦唔會有機會識字同寫野嘅人,就攞住諺文黎寫日記。
但冇幾耐,諺文就畀人$_{\text{ban}}$咗。

朝鮮國嘅第十任國王燕山君,係一個大暴君,專搞埋晒啲寸斬、炮烙、拆胸、碎骨飄風等等嘅酷刑。而且佢又淫蕩非常,後宮膨脹,又時不時將啲佛寺改建成為妓院。百姓民怨沸騰,淨係得把口就用諺文寫野侮辱同詛咒佢。燕山君就下令取締諺文。而自此之後,諺文都主要淨係喺婦女同僧侶之間流傳使用,故諺文亦稱為「女書」或者「僧字」。

呢個情形一直到十九世紀末、二十世紀初朝鮮半島民族意識強烈提升先致開始有改變。大韓帝國高宗國王喺1894年至1896年間推動甲午改革,其中一部分頒布命令規定「法律條文與公文基本上應採用諺文;但全漢字或漢字與諺文混用的版本於必要時可以增加」。之後冇幾耐日本帝國夾硬吞併韓國嗰陣,日本人頒發咗《朝鮮教育令,規定埋一個星期中韓語同諺文嘅教育時數》(但韓語同諺文係冇官方地位),亦編製咗「詞根用漢字,虛詞用諺文」嘅韓文教科書。但係到第二次世界大戰開戰,諺文就被視為係朝鮮國族主義嘅象徵,又事被禁止使用。

諺文嘅下一個歷程碑,係1948年政府提出嘅《諺文專用法》。自此,韓漢混寫就真係抬頭。到咗六七十年代,極力主張使用諺文嘅總統朴正熙,喺1970年發表漢字廢止宣言,小學冚把爛廢除漢字教育。到咗1980年代中期,韓國嘅報紙、雜誌等,開始逐漸降低漢字使用頻率。噉係因為幾乎冇接受漢字教育嘅世代(諺文世代)開始佔多數,搞到漢字嘅出版物賣唔出——漢字續漸安樂死。

與此同時,北韓做咗自己嘅諺文拼寫改革,亦一刀切完全廢除晒所有漢字;中國改用簡體字,二簡字唔成功要打倒褪;星加坡曾經試過自己簡化漢字,但用用下就直接採用中國嘅簡體字方案快靚正;台灣、香港、澳門就繼續用繁體字;越南完全廢除漢字,採用以法文為參考基礎嘅拉丁拼音文字發明咗「𡨸國語」(chữ
Quốc ngữ);而日本就繼續漢字假名混用。

而到咗近年,漢字教育喺南韓仍然係教育政策嘅一大爭議。有人認為廢除漢字造成嚴重文化斷裂,搞到韓國人連《世宗原詔》同自己嘅憲法都睇唔明。廢除漢字亦導致韓語無法攞自己嘅漢語詞根發展新詞彙,社會同經濟發展所需要嘅新詞彙只可以全部透過音譯英文呢種嘅烏呢媽叉手段黎解決。唔識漢字,亦令韓國同中國、日本、台灣等地文化交流上出現隔膜,甚至因為同音字問題而搞出「賤出名將事件」。但亦有人認為,漢字「三多五難」,而且而家諺文已經完全成熟,學漢字根本就多餘同嘥時間。況且,諺文咁鬼犀利,咁鬼精準,點解要學人地國家(中國)嘅野?最大力反對漢字教育嘅「韓字學會」甚至話「韓字係可以喺全世界面炫耀嘅科學文字」、「漢字係特權階層嘅反民主文字」、「根本就冇南韓國民認為韓字專用唔方便」。呢種嘅心態同朝鮮民族主義結合,都滋生咗唔少語出驚人嘅說話,譬如咩「韓國之所以可以科技發展一飛沖天係因為諺文奠定咗韓國嘅數學同邏輯基礎」。到咗而家,漢字復用喺南韓仲係處於一個唔嗲唔吊嘅狀態。

\subsection*{諺文畀粵語嘅啟示}

我哋而家嘅粵語,仍然處於一個未能夠全面「我手寫我口」嘅落後境況。「本字考」仍然係處理粵語「有音冇字」嘅主流方案,以拉丁字母為基礎文字嘅粵拼亦有其推廣——但呢啲嘅手法其實都係自己嘅問題同盲點。「本字考」嘅基礎理論同方法論其實非常可疑,生產解決方案龜速,而且毫無系統,民眾參與唔到。而且本子考完全係事後解決主義,係社會度有咁上下數量嘅人用一個詞,我哋先至會搵個本字出黎,故此追唔上民間粵語嘅高速發展,甚至係排斥自然發展。其實呢度已經見到本字考嘅方法論謬誤——如果個詞係新嘅,《康熙字典》裏面又點會有本字呢?咁樣唔係刻舟求劍係乜野?所以話呢,本字考其實係延續住粵語言文分離嘅劣況,雖然有陣時佢哋都生產到啲雅味不俗嘅方案,譬如「齮齕」(gee
gut)、「䒐䒏」(忟憎),同「𪘲牙聳䚗」(依牙鬆鋼)噉,都咪話唔話有聯綿詞嘅feel。


至於拉丁拼音方案,字型美感上同漢字相斥,即使全民識用,都冇人會當作為正式文字。除非我哋用極其粗暴極權嘅手段全面廢除漢字,將漢文粵語完全消滅打殘,將粵文構建成果推倒重來,喺呢個一片荒蕪嘅曠野度畀粵語全面採用粵拼,否則粵拼就只會係類似普通話拼音嘅輔佐子系統。外國人或者香港嘅少數族裔學咗粵拼,其實都會依舊係文盲,因為根本冇野係用粵拼寫。

韓國廢除咗漢字,其實可以話係個陰差陽錯嘅偶然,而唔係佢演化嘅必然結果。漢字好似已經喺韓國徹底死亡,但其實不然,佢仍然有復活嘅機會——只要政治環境風向改變,基本上係必定會復活,因為好似崔萬理嘅士大夫喺韓國依然存在。漢韓混用,反而先至係最自然同效益最大化嘅方案。所以,我哋應該從諺文歷史度專注嘅,唔係漢字嘅死亡,而係諺文點樣完成咗「韓語有韓文」個工程。更重要嘅,廢除漢字然後諺文專用所損失嘅,係比唔上唔用諺文夾硬用漢字寫韓文嘅荒謬。

「{\koreanfont 감사합니다}」當然唔夠「感謝{\koreanfont 합니다}」咁多資訊啦;「{\koreanfont 안녕하세요}」(「安寧{\koreanfont  하세요}」)、「{\koreanfont  죄송합니다}」(「罪悚{\koreanfont 합니다}」)、「{\koreanfont 미안합니다}」(「未安{\koreanfont 합니다}」)、「{\koreanfont  실례합니다}」(「失禮{\koreanfont 합니다}」)等等嘅例子都睇到漢字嘅表意同跨語言溝通功效啦。但係我哋要對比嘅唔係「諺文專用」同「漢諺混用」,而係「諺文專用」同「漢字專用」。如果呢乍野全部用漢字寫,噉啲「habnida」點算呢?用「合尼大」假借黎寫?如果用漢字假借黎寫韓文會覺得係篤眼篤鼻嘅,噉點解「多謝曬」裏面嘅「曬」我哋又唔覺得係問題?呢個「曬」,無論你係用「曬」又好,「晒」又好,個詞都係同個漢字冇意思。你用漢字,反而係隱藏同模糊咗粵語嘅語法部件。同樣道理,「做咗」「做緊」「做埋」「做過」「做住」「做親」「做做下」嘅「咗、緊、埋、過、住、親、下」其實全部都係攞咗漢字黎做啲漢字唔應該做嘅野。

更重要嘅係,我哋香港人因為我哋嘅政治成見同我哋引以為傲嘅文化背景,好容易無視咗漢字教育,真係需要巨大成本。要民眾學漢字,你係要投入大量資源同事件架。雖然我哋會覺得喺依家呢個世代,呢啲錢同事件根本唔係啲乜野。但係諺文發明咗之後幾百年黎都冇政府支撐,都可以喺低下階層繼續傳承發展,反而漢字就無法擴張佢自己嘅領域,就已經暗示住邊一個嘅成本效益比較好。我哋因為愛戀漢字,所以抗拒所有漢字以外可能可以解決到我哋語言寫唔出嘅方案。某程度上,我哋係寧願保住漢字,粵語唔可以原汁原味我手寫我口都冇所謂。再簡單啲黎講,我哋個個都係崔萬理。

廣東話,配有自己嘅文字。粵語,配有自己嘅文字。我哋,配有自己嘅文字。
所以,要粵字改革。

央乙·止子·丩丐·丩百·亾乇·禾会·
亾丐·夫丂·〡〧〩·乃千·〡〇·央乙·〩·央乜·




\chapter{粵切字可「有邊讀邊」,毋需教育局,就可無師自通}

「粵切字」,又稱「粵砌字」,最犀利嘅·夫子·此居,就係佢可以啟動激活到香港人「有邊讀邊」嘅閱讀漢字能力,加上透過上文下理推理同我哋對既定詞彙格式嘅把握,基本上係可以一目瞭然讀得明。

呢個點係點解粵切$_{easily}$比所有其他拼音方案優勝嘅原因,包括諺文、假名,甚至任何一個拉丁拼音。佢哋全部都要教,但粵切字基本上唔太使:要識讀基本上唔識教,要識寫就當然要背字符——但係,你唔鍾意一個字符,都可以換。只要「反切」嘅拼音原則喺度,就唔使太擔心字符點樣變。「變態假名」幾百個,底層嘅原則係萬變不離其宗。

時粵切字,可以繞過官方教育機構,唔使去攞呢個所有人都認為必須攞到先至可以展開語文構建工程嘅體制嘅特質,係一樣就連「粵拼」都未必做到嘅嘢。所以,所有話必須攞到政權先好搞語文改革嘅說話,都係錯誤。粵切字,就係一個唔使去奪權就可以架空現時語文政權嘅方案。

而粵切字做到有邊讀邊,意味住佢可以話係我哋現時「粵製漢字」嘅進化品。與現時「粵製漢字」,或者「粵方言字」唔同嘅係,粵砌字係有系統,可以有規則地預測、組裝、延伸。唔似而家雜亂無章、不成系統,只有局內人識睇識用,局外人就一頭霧水。

粵切字,係我哋應該攞黎用嘅自然文字系統。粵切字,就係未來嘅粵字。


\chapter{點解漢字專用粵文對廣東話黎講係慢性自殺}
乜野係「漢字專用」?「漢字專用」就係淨係用漢字黎寫野。乜野係「粵文」?「粵文」就係將粵語口語完全透過文字表現於書面嘅文。咁,「漢字專用粵文」,就係淨係用漢字黎將廣東話口語完全表現於書面嘅文。

我哋而家大部份嘅粵文都係漢字專用,譬如《迴響》、蘋果,同啲$_{youtube}$字幕嘅粵文,都係漢字專用。

\subsection*{多語漢字專用嘅集體博弈不穩定}
如果廣東話以漢字專用嘅模式去書寫粵文嘅話,廣東話呢種嘅語言就會慢慢死亡。點解?因為廣東話呢者口講嘅語言,就會因為佢嘅書面語,係淨係用漢字寫。而喺一個時空之內,世間上最到只可以有一隻語言嘅書面語係漢字專用。如果世界上有多個一隻語言係漢字專用緊,就會構成唔穩定。而呢一個唔穩定嘅情況,會好快陷入競爭動態,到得返一個留低嗰陣先至會停止,回歸穩定。

\subsection*{漢字學習成本極高,係要學嘅話就一定係學利益最高嘅}
人要考慮邊隻語言去學或者用嗰陣,我哋係無法避免利益行頭嘅成本效益衡量分析。大部分人係唔會脫離到市儈嘅效益運算。人學語言,基本上籠統而言,就梗係盡求以最少嘅成本同負擔,就可及換取到最大嘅利益。

而好多時候,一個人去學習同使用某一隻語言嘅成本,係同嗰隻語言嘅文字掛鉤。如果兩者語言都係用同一隻文字,咁佢哋就會有一樣嘅文字學習成本。而對於用開其他文字嘅人黎講,兩隻都係用漢字嘅語言,學習成本會係差唔多——都係圍繞住漢字嘅學習成本黎計數。

\subsection*{漢字專用,客源重疊}

漢字嘅學習成本,係非常高。而個使用漢字到能夠存取利益嘅門檻,亦係非常高。基本上所有漢字專用嘅語言,都係面對住同樣嘅情況:學習成本高,啟鎖取利路途長、所需累積語言能力高。既然係咁,唔去學預計效益最高嘅語言嘅機會成本效益比就會大到難以承受—— 而大部分以利行頭嘅人,睇到呢一點,選擇已定。佢哋就必定係會去學效益最大嘅漢字專用語言,其他嘅都置之不理。而而家效益最大嘅漢字專用語言,係普通話,而唔係廣東話。所以,如果廣東話同普通話都一樣係漢字專用,廣東話嘅客源就必定輸畀普通話。

\subsection*{漢字學習成本極高,學完一隻漢字專用語言,難會再砌另外一隻}
雖然學咗一隻再學另外一隻,譬如學完普通話再學廣東話咁,因為你已經把握咗漢字,所以學第二隻漢字專用語言嗰陣可以慳返唔少成本。但係所慳返嘅同所賺到嘅比例唔係吸引得拒無可拒,加上唔係話個個都咁有魄力去不停咁學,好大機會出現嘅情況就係一百個人學咗一隻漢字專用語言嘅人裏面,之後再去學另外一隻嘅只有極度之少。姐係話,一旦一個人去咗學普通話之後,你想佢再兜返轉頭學廣東話?想創你個心。最緊要嘅係,所有漢字專用嘅語言都純粹因為佢哋係漢字專用而係死敵。佢哋因為自己係漢字專用,而逼使咗大家互相嘅客源為同一班人。而廣東話,係因為佢嘅政治同經濟現實,唔會鬥得贏普通話架。

\subsection*{中產:見利遷語,語言不忠}
仲有,以上嘅論點同分析,唔係淨係適用於外國人,而係粵語母語人士都適用。粵語為母語嘅父母,只要語言忠誠度稍微略低,比較容易見利忘義,就會表現曬上述嘅「棄粵追普」現象。加上嗰種完全無知嘅「咪都係用中文字!咪都係中文!」嘅思維,粵語為母語嘅父母就係將自己嘅子女送曬去學普通話,廣東話就係「無所謂啦」處理作罷。你見到而家幾多中產家庭由細路出世開始就淨係講英文,就已經可見端倪。粵語同普通話直接喺同一個文字體系度競爭,情況就更加差。廣東話,就會咁樣喺漢字專用嘅語文秩序度,監生被棋局餓死。一山不能藏二虎嘅規則顯然易見。漢字專用嘅語言,只可以有一個。

\subsection*{要漢字專用,就要坐穩「中文」嘅帝位}
呢個亦意味,如果你想廣東話係漢字專用,你就必須使所有其他都係漢字專用語言,一係就停止漢字專用改為「漢甲混用」,一係就吞噬佢哋。要漢字專用,你就要變成「中文」。要變成「中文」,你就要將所有其他嘅變成你嘅方言。

廣東話要漢字專用,就要殺死所有其他漢字專用嘅語言。

\subsection*{「阿乃椅子羽個乃教師二有馬下」—— 如果日文係漢字專用,你學得黎?}
曾幾何時,日本、朝鮮、越南,都係以漢字專用嘅語文秩序黎書寫佢哋嘅語言。日文嘅《萬葉集》就係咁樣寫「阿乃椅子羽個乃教師二有馬下」(昨日、あの椅子はこの教室にありました。)寫日文,論盡程度有而家粵文嘅漢字專用過之而無不及。

到咗而家,日本已經係假名漢字混用,韓國就諺文專用,而越南就係已經全盤拉丁化。因為漢字專用而語言消亡畀中國嘅中文吞噬嘅可能性因此而減。試問,如果佢哋仲係用漢字專用緊,論論盡盡咁去寫佢哋嘅語言(如圖),擺喺普通話隔離,你仲會唔會咁容易有心力去學佢哋?但係中國內嘅諸夏語言幾乎個個都仲係用漢字。上海嘅上海話、潮州嘅潮州話、惠州嘅客家話,安徽嘅安徽話,南昌嘅贛語,仲有香港嘅廣東話,通通都係用漢字專用。喺咁嘅情況下,你有咩誘因去犧牲學習普通話嘅機會,轉移去學其他嘅漢字專用語言?何來划算?答案就係:一啲都唔划算。而唔划算嘅後果就係你漢字專用嘅語言慢慢被淘汰,直至你迷失於歷史長河之中。

上述嘅演化機理已經喺香港展開咗好幾年。黎香港讀書嘅外國學生冇幾個係會學廣東話,甚至更加會因為已經學咗普通話,加上漢字同中文學術界嘅不誠實宣稱廣東話係中文方言(試問「話」點可能係「文」嘅「方言」?),個個都不屑學廣東話,不屑融入香港本地文化,甚至厭惡我哋此類嘅要求,倒返過黎發狼戾。我哋自己嘅中產,就更加係表現咗語言不忠嘅極致。佢哋攞住英文喺世界各地橫羅語利,講廣東話嘅普通階層就繼續捱廣東話嘅貧窮。廣東話,變緊一隻窮人同$_{non-stakeholder}$嘅語言。

\subsection*{漢字專用係自殺,粵漢混用可奔日韓}
漢字專用,係慢性自殺。如果廣東話要有希望可以逃出生天,要成為好似日文、韓文、越南文、甚至係好似英文、法文、德文咁偉大嘅語言,我哋就必須放棄漢字專用,放棄對漢字嘅迷戀。粵語配有繼續生存落去發展嘅天直—— 我哋配有自己嘅文字。粵切字,就係一個可以畀粵語直奔日韓嘅文字系統。粵切字,故此,應該成為粵字。粵切字,就係等住成為我哋未來嘅粵字。


% 個々都係「漢字專用」,都係「中文」,咁點解係要學你嗰隻?




% 壯語	韓語*	越南語	日語	天津話	客家話	福建話	贛語	上海話	粤語	「中文」
% 等
% 。
% 伝
% 𠷯
% 理
% 性
% 𪝈
% 良
% 心
% ,
% 應
% 當
% 待
% 㑣
% 彳    比
% 彳   農
% 一
% 樣
% 。	佈
% 佈
% 𧾷馬
%  丁 刂
% 𨑜
% 
% 就
% 𠷯
% 自
% 由
% ,
% 尊
% 嚴
% 𪝈
% 權
% 利
% 佈
% 佈
% 平	愛
% 厶
% 精
% 神
% 〇 乙
% 又
% 行
% 動
% 丷
% 余
% 𠃌
% 丷 万
% 夕
% 。	理
% 性
% 果
% 良
% 心
% 乙 
% 賦
% 與
% 馬 加
% 牙 叱
% 乙
% 〇
% 厼
% 覀
% 又
% 兄
% 弟	果
% 權
% 利
% 厂
% ㇏ 叱
% 仒
% 同
% 等
% 下
% 夕
% 。
% 人
% 間
% 𠃍
% 天
% 賦
% 的
% 乙 〇
% 又	毛
% 木
% 人
% 間
% 𠃍
% 他 厂
% 仒
% 那 乙
% 夕夕 厂
% 部 
% 他 仒
% 自
% 由
% 又
% 于
% 厼
% 可 乙
% 尊
% 嚴	処
% 𠇍
% 饒
% 冲
% 情
% 英
% 㛪
% 。	。
% 每
% 𡥵
% 𠊛
% 調
% 特
% 造
% 化
% 班
% 朱
% 理
% 致
% 吧
% 良
% 心
% 吧
% 勤
% 沛
% 対	畢
% 哿
% 每
% 𠊚
% 生
% 罖出
% 調
% 特
% 自
% 由
% 吧
% 平
% 等
% 𧗱
% 人
% 品
% 吧
% 權
% 利	寸部天乃人間波、生礼奈加良仁之天自由天安利、加川、尊厳止権利止仁川以天平等天安呂。人間波、理性止良心止遠授計良礼天於利、互以仁同胞乃精神遠毛川天行動之奈計礼波奈良奈以。	人個頂個生而自由,在尊嚴和權利上般兒般兒大。他 們趁理性和良心,並應以兄弟關係的精神相對待。	人人生而自由,在尊嚴同權利上一律平等。佢丁人賦有理性同好心田,並應以兄弟關係個精神相對待。	人人生而自由,在尊嚴合權利上一律平等。因賦有脾胃合道行,並著以兄弟關係的精神相對待。
% 	人人生而自由,在志向跟權利上一律平等。渠們賦有理性跟良心,並理當以弟兄義氣相對待。
% 	人人生而自由,拉尊嚴脫仔權利上一律平等。伊拉有理性脫仔良心,並應以兄弟關係個精神相對待。
% 	人人生而自由,喺尊嚴同埋權利上一律平等。佢哋有理性同埋良心,而且應當以兄弟關係嘅精神相對待。	人人生而自由,在尊嚴和權利上一律平等。 他們賦有理性和良心,並應以兄弟 關係的精神相對待。
% *韓文旡漢字專用例子,係諺文字母去其「口訣」符號旡對應,再繼而併合而成。
%  


\chapter{粵文一日用口字旁,粵語就一日係方言}

「一啲」、「呢啲」、「嗰啲」、「邊啲」、「啲野」、「快啲」、「少啲」、「呢啲」、「遲啲」、「 鬆啲」、「郁啲」、「一啲啲」、「住好啲」、「食好啲」,同「講呢啲」。

「啲」呢個詞,根據《粵典》,有兩個用途。第一個係解「少少」,通常放喺形容詞後面用嚟做比較。而第二個,就係表達複數,用作為一個不可數嘅量詞。唔難睇得出,呢個詞係一個具有重要語法地位嘅詞。
「啲」呢個字嘅結構,由「的」同「口」合併組成。「的」係聲旁,負責表聲,而個口字旁,就係意符,負責暗示或者指導個意思出黎。所以,「啲」係個形聲字。再嚴謹少少黎講,個口字旁所負嘅責,就係\scalebox{0.5}[1.0]{扌}\scalebox{0.5}[1.0]{󰖖}低話呢個詞係口語用詞,係廣東話,係口語。
呢種咁樣嘅「口語詞」喺粵文好多,但唔係粵文獨有,東北話書面語、大眾普通話、吳文都有唔少。但係天下間五花八門嘅「口語字」,喺佢哋所出現嘅文章,都有同一個隱性效應—— 就係佢無時無刻都竊竊私語緊喺度同讀者講:「呢篇文章所寫嘅語言,根本就係方言」。
「口語字」,佢自己嘅本質,就係要表達口語。佢嘅運作,就係攞唔係口語字嘅字,借黎標音。而為咗我哋可以輕易地睇得出佢淨係攞黎表音而唔要個字本身嘅含意,我哋就加個口字旁,加以強調,以資閱讀。姐係話,你咁樣做唔係用漢字嘅表意功能啦,你本身寫嘅嘢唔係正統雅言啦—— 雅言要寫出來,點可能要用埋呢啲咁嘅同低莊手段嘅啫?齋睇一篇成篇都係口字旁嘅廣東話文章,同一篇全部都係正宗有晒甲骨文祖先嘅白話文,就知道邊一個份量深厚嘅古老雅言,邊一個係要借錢渡日嘅小方言啦。一篇文章,睇佢用嘅字,就可以知道邊一個係可以擔起文化大旗嘅嚴肅文學,邊一個係冇料扮四條兼玩玩下嘅方言文章。你嘅嗰啲「口字旁字」,個個都係用傳統漢字加個口字旁上去,仲唔係方言?乜你哋班野唔係心知肚明呢個原因,先至呃鬼去話「尐」先至係「啲」嘅正字咩?
聽到呢啲說話,我哋當然會好嬲。但係往往啲回應都係啲呃細路、經唔起深究、而且策略上其實懵盛盛煮重自己米嘅回應:比如啲咩「廣東話係唐宋雅言論」同咩「『攰』喺《說文》度有」嘅論調,同埋嗰啲本字考。講到尾,你咪仲係喺度「以華度己」?你哋想扮係中華正統,「咁噉呢嚟嘢咩哋嘅啦喇囉喎呀嘩啱㗎」一大咋亂七八糟嘅口字旁,呃得到邊個?根本就係太監扮皇帝。你就連個「啲」字呀,都係用官話嘅「的」作本位加個口字旁㗎咋。「的」字喺廣東話根本就唔係讀「󰦦」,係你哋班嘢唔知頭唔知路,將中文裏面最為重要嘅介詞字不問自取,然之後加個口字旁就當自己乜乜七七,笑死人咩。「有音無字」,係方言嘅特徵。

思哲誠實嘅人唔會對呢番說話駁嘴,因為思哲誠實嘅人係會自己同自己講呢番說話—— 按照佢個範式黎思考,咁樣嘅結論就係必然。

而我哋要嘅,就係跳出呢個範式,要挑戰呢個範式,要拒絕呢個範式。

乜你忍受到「啲」呢個喺廣東話裏面有舉足輕重語法地位嘅詞以普通話嘅「的」加個口字旁嘅方式存在落去咩?乜你忍受到廣東話嘅書面語視覺上呈現住「我哋係普通話嘅變體」係信息咩?乜你忍受到漢字對廣東話喺書面上宣判為方言嘅呢種對待咩?乜你忍受到廣東話嘅「啲」以普通話嘅「的」加個口字旁就算?

如果我哋繼續口字旁落去,廣東話就永遠都只會係一隻文字上用漢字 B 隊嘅一種 B 系語言。佢永遠都會係排第二冇得當家作主,仲要永永遠遠以其他人去量度同定義自己,而唔可以自己定義自己。粵語喺呢一個咁樣嘅遊戲度,可能可以做到二房,至多可以做到正室。睇埋家下時局個樣,你做到二奶小三就應該・劏豬還神喇,再唔係連妹仔都冇得你做,等死啦。

但係,我哋要嘅,唔係廣東話做正室。我哋要嘅,係廣東話自己當家作主,唔駛寄人籬下,唔駛睇人面色聽人說話。我哋要嘅,係廣東話好似日文韓文咁自己可以堂堂正正話自己就係語言。我哋要嘅,唔係要說服人嘅依據或者論調。我哋要嘅,係擺晒喺你面前,你冇得否認嘅事實。我哋要嘅,係一隻可以畀到我哋可以好似日語韓語咁唔會有人覺得呢隻語言只不過係中文方言嘅文字。我哋要嘅,係一隻會令到所有嘗試咁樣諗嘅人腦短路嘅文字。我哋要嘅,係一隻可以畀到廣東話尊嚴嘅文字。

粵語配有自己嘅文字,因為粵語應予有尊嚴。

\chapter{雜論1}
有好多人成日會揶揄粵字改革,話咩不如用國際標音仲好過啦。問題係文字唔係淨係標音系統。所有嘅拼音文字都係有表意性嘅,而表意性就係透過特定同有限嘅不規則拼音表示。同時間,絕對無誤嘅表音係會造成書寫同閱讀嘅嚴重障礙(香港人連 the 同 da 都分唔開,會分得開 p ph p’ 咩?)。但係最重要嘅係,文字唔係淨係攞黎表達口語,亦唔係淨係攞黎溝通。佢有好多好多嘅共用同職責,而其中一個就係賦予佢嘅用家群一個自己隸屬嘅群體,仲有與其相㨢嘅尊嚴。粵切字為粵語賦予尊嚴,務求可以好似諺文咁為韓語賦予尊嚴。呢個,亦係點解拉丁化乃為下策嘅原因。
話時話:如果有人真係蠢到用IPA黎寫自己嘅語言嘅話,好快就會用爛。因為群眾會頂唔順啲規則然後迅速簡化同大產特產例外用法,最後喪失最初嘅最大優勢:其百分百精準嘅表音能力。

\chapter{雜論2}
康德講過,上帝淨係畀咗兩個手段畀人類去建立自然嘅獨立群體,一個就係語言,另外一個就係信仰。而歷史話畀我哋聽,語言係通常都冇辦法喺異鄉度維持到超過三代。睇睇移民美國嘅愛爾蘭人,德國人,甚至啲ABC,就知。如果語言瓦解,個群體就會失去生命力,慢性死亡。語言唔夠掂,就要有宗教。香港人的確要有信仰,仲要有自己嘅(神)聖經(文)。猶太人嘅摩西五卷,就係佢哋文明嘅精髓。希伯來文曾幾何時已經變成咗死語言,以色列復國嗰陣就係靠摩西五經及其環繞嘅相關文獻重建、復活、同現代化佢哋嘅國語:希伯來文。我哋都要咁樣做。而係呢個浩瀚嘅工程裏面,粵切字一定要有地位。粵語配有自己嘅文字 ,而粵切字就係粵語應用嘅文字。

\chapter{雜論3}
官話地區基本上完全毫無髮損。粵語仲一枝公撐緊。吳語就已經死曬。粵切字,好大程度上只係文字改革嘅開始。官話必須有多個文字改革,全部砲彈用曬佢。拉丁化、阿拉伯化、改造諺文、自家發明拼音,全部都要派上場。吳語小字同粵切字已經照顧好吳語同粵語。女書必須改革同活化,參考日本假名繼而系統化為湘、贛提供拼音文字。至於客家話同閩南話,就睇班台灣人可唔可以疊埋心水推白話字或者佢地啲注音符號喇。

\chapter{雜論4}
粵切字係邊個發明,根本唔重要,亦唔應該去而家專研。假名係邊個發明,係空海定係啲平安時代嘅貴族女人,根本唔重要。諺文到底係世宗單人匹馬發明定係集賢寺眾人合力砌出黎,根本唔重要。漢字係倉頡定係伏羲氏整出黎,唔重要。到底係邊一條躝癱將$_{Phoenician}$借黎用寫做$_{Greek}$,邊一條友將希臘字母有借冇還變咗做拉丁字母,唔重要。我哋要嘅,淨係粵切字嘅流傳開去。因為我哋相信嘅就係,粵切字嘅勝利就係粵語嘅勝利。所以,所有睇緊呢個page嘅文青同文創家,請你哋狠狠咁強我哋嘅舖,用我哋嘅輸入法,去寫詩寫文寫小說做二次創作。你想做崔萬世鬧我哋嘅話都得。我哋最怕嘅,就係淪為$_{con lang}$,姐係畀人當係玩泥沙。所以,請大家大搶特搶。最好係當粵切字係石頭度爆出黎嘅野,係人人可以攞黎當係自己嘅野。我哋需要咁樣。我哋必須要咁樣。



\chapter{雜論5}
想了一下關於粵切字推廣和文學構建的策略。

將粵謳用粵切字(部分用家喜稱粵砌字)轉寫來得出非漢字的本土粵語文學,有一個很大的問題,就是這樣的轉寫文學是假的。如果粵切字文學代表著「真正」的粵性,那粵切改寫的粵謳所散發的「粵性」就是假的。

我們視萬葉集為日本文學的開端,但十九世紀的日本國族構建者卻覺得萬葉集的全漢字性還是讓他們的文化地位非常尷尬,反而由女人寫,全假名或幾乎全假名的《源氏物語》就沒有這一種的漢字尷尬了。

用粵切字改寫粵謳,目的無非在於生產粵語《源氏物語》。但粵謳的本質是《萬葉集》。以粵切字改寫粵謳以得出粵語《源氏物語》來提高粵切字或粵謳,我恐怕,會為粵切字惹來非常強烈的義責。粵切字可以為粵謳標音,但改寫恐怕是大逆不道,是製造偽書。

如果要有粵語(粵切字)的《源氏物語》,真的只能製造,不能作偽書。方法就是讓小學生和中學生用粵切字書寫。女中學生是最重要的瞄準群。

其實,所有在諸夏中嘗試從非政府層面推廣漢型拼音文字(即美感上與漢字相容/可以與漢字混用的拼音文字),瞄準年輕女性,都是策略上策。低下階層、海外僑民$_{ABC}$、本地外族($_{think}$ 香港的南亞裔,大陸人,鬼佬)都是推廣焦點。

女性在這一種的社會文字改革有巨大的策略價值。基本上搬西方的基本女權主義理論就可見端倪。而歷史上,我們也看到,女性所面對的壓迫或困難,是驅使他們遠離漢字的動力。假名是由日本女性貴族發明的;諺文反對者全部都是男性士大夫,而燕山君上台後血腥鎮壓諺文,而諺文就透過低下階層的女性保存;永江的女書就不用說了。
你可能會認為今天的女性沒有古代般受壓迫。的確是的,當時壓迫還是存在。這樣就夠了。你想想潮州家庭的那種重男輕女,還有傳統華人家庭的那種完全沒有私隱的變態,就可想像到香港的中學女孩子,會用粵切字來寫日記、短訊、愛情小說—粵切字的《源氏物語》就可面世了。
向外族人推廣粵切字是$_{capstone}$,不能是中間的過程。如果以為可以透過向南亞裔或鬼佬或大陸人推廣粵切字而讓粵切字變成為粵語的未來國書,是大錯特錯。這樣做,粵切字就會馬上被打成為外來入侵的文字,$_{blahblahblah}$ 你已經可以想像到連登和高登那些連梁天琦也要罵是支那人的傻瓜和依靠中文貴族秩序搵飯食的士大夫一定罵死—而他們會成功牽動到輿論—那粵切字就必定石沈大海永不超生了。向外族人推廣,是最後一步。
當然,粵切字的例外一個構建自己legitimacy 文化合義性的渠道,就是出面的抗爭。只要街上有一個塗鴉寫著「古王.夫玉.亾丈.丩王,厶子.大丐.丩百.文丁;亾丈·丩王·大玉·力甲,禾兮·央乜·此𥘅.力冇」,我就已經贏了。
以上的策略考慮,理應都適用於吳語小字。


\chapter{雜論6}
排斥係違反物理架。問題係粵語根本變緊中文,所以我哋嘅粵語細胞衍生咗可以畀普通話病毒依附感染嘅蛋白質。大陸詞彙湧入,係因為粵語出現真空,秩序較強嘅語言就湧入填補,唔可以就咁歸咎曬於政府。反而你地嘅中文老師,姐係所謂嘅士大夫,先至係罪魁禍首。你地要做嘅,係喺日常生活裏面毫不顧面子咁發明新詞。有人用士大夫嘅措辭黎質疑反感發狼戾你就繼續用,用到佢地吐血為止。「人地唔明」係一個白癡嘅辯駁。約定俗成先至係上帝。

\chapter{粵切字嘅下一步推廣策略係乜嘢?}
粵切字成功嘅唯一可行策略,必定係一個行行下就會自己消滅「推行」嘅需要。姐係要去中心化,落之群眾,輕「公佈」重「流佈」。因為咁樣粵切字先至會有有機性。
要激活同啟動到呢種嘅散播模式,首先要粵切字嘅文字基建搞掂先。(注意:係文字基建,唔係語文基建。我哋淨係需要·忄臼旡.你寫唔寫到出黎,我哋唔·忄臼旡.你點寫)。
而家已經有輸入法同字體,已經成功咗多過一半。但係問題係已組裝嘅粵切字淨係可以喺$_{word doc}$度出現。網上面冇辦法讀取到呢啲字,至多只可以用啲未組裝嘅粵切字黎寫。咁樣就大大侷限咗我哋流通出去嘅可能性。解決呢個問題係我哋嘅首要任務。
點搞?最簡單最快嘅方法就係整兩個$_{Google chrome extension}$,(1)一個·大万·力冇·咗之後就可以讀取到曬啲組裝粵切字,(2)另外一個裝咗之後就可以直接喺個$_{Google chrome}$ 瀏覽器度打粵切字。(1)重要性遠超於(2)。
呢個我哋已經做緊。但係極度唔夠人。如果你想幫手,敬請兼且歡迎毛遂自薦。
呢樣野搞掂咗之後相信使用人口就會大增。咁之後嘅策略,就係要鼓勵產生粵切字嘅《源氏物語》《萬葉集》《奧德賽》《理想國》同《形上學》。粵語唔可以再·疒厶乜.疒夫乍.無詩無文無哲無數嘅呢種落後局面。仲要重譯《聖經》。

\chapter{我地必須虛心懺悔反省}
我地必須虛心懺悔反省,點解講粵語嘅人,咁多都係思哲不全,言無黹語不法,氣如蠻夷,思緒污穢。

\chapter{「粵語入文」係一個極度自我鄙視嘅思維模式}
其實,「粵語入文」係一個極度自我鄙視嘅思維模式。佢其實就係文字「官話作主粵作客」秩序嘅呈現,亦係白言文對廣東話嘅根本性歧視同排斥嘅運作機理。
《迴響》裏面有人講過,有「粵語入文」,咁係咪都有「普通話入文」同「英文入文」?
我哋要嘅,唔係咩粵語入文。我哋要嘅,唔係做二房。我哋要嘅粵文,係粵語白話文運動,係文化獨立,係世宗路線。呢個,就係我哋嘅願景,亦係我哋嘅責任。

乜你忍受到「啲」呢個喺廣東話裏面有舉足輕重語法地位嘅詞以普通話嘅「的」加個口字旁嘅方式存在落去咩?乜你忍受到廣東話嘅書面語視覺上呈現住「我哋係普通話嘅變體」係信息咩?乜你忍受到漢字對廣東話喺書面上宣判為方言嘅呢種對待咩?粵語配有自己嘅文字,因為粵語係隻有尊嚴嘅語言。


\chapter{拉丁化係粵切字嘗試避免嘅命運}

簡單講講粵語文字秩序發展所面對嘅棋局。按照現時嘅發展,最大嘅兩個玩家就係純漢字粵文同粵語拉丁文。而粵切字基本上係不成氣候,競爭地位同粵語諺文同粵語假名一樣不入流。而因為基本上粵語諺文同粵語假名基本上冇可能發展出去,所以發展潛能粵切字稍微比諺文同假名好少少。

純漢字粵文基本上同本字考嘅關繫係曖昧不清。本字考勝出嘅機會係0,但係佢可以喺純漢字粵文呢度嗰度·氵厶今.啲落去咁苟且偷生。呢點冇咩好爭議,因為而家就已經係咁樣發展緊。但可惜嘅係,純漢字粵文係慢性自殺。過去一年嘅事已經將自殺速度大大加速,整個粵語生態圈會喺20年內完全崩潰,香港會上海化。

喺呢種文化被清洗咗嘅情況下,就好似真空必須會由空氣填滿一樣,拉丁粵文就會立刻暴發,就好似啲菇菌喺朽木度滋生咁。

注意,拉丁粵文迅速自然填補真空嘅前提,係有一大乍識講廣東話但係零漢字能力嘅人。呢個前提係唔確定嘅。我哋面對更加有可能嘅情況係粵語滅亡,普通話大行其道嘅情況。

基本上,我哋而家係唔會有粵拼文化出現。我哋太過漢字本位喇。要喺而家呢個情況度推粵拼,仲要有受整個群體認可嘅粵拼文學產品出台,根本就係冇可能。八間大學嘅士大夫全部群起而攻之啊陰功。

姐係話,拉丁粵文唯一可以出台稱霸嘅時機就係光復之後。如果光復之後你繼承嘅語言生態圈係仍然有粵語群同有文化資本生產力嘅粵語漢字使用家,你以粵拼作為粵語嘅文字系統呢個舉措就係打爛自己飯碗。如果你唔廢除漢字,羅漢並用,歷史話比我哋聽(1)呢條路係唔穩定,幾乎一定變成為拉丁專用,(2)你透過你自己文化所賺嘅錢一定會相對於現實下降。拉丁化嘅$_{best case scenario}$ 係越南(注意:係 $_{best case}$。呢個就係天花板),差啲嘅就係台語、客家語、壯語、彝語呢啲拉丁化咗嘅語文。你死又死唔去,但係又冇文化資本,毫無吸引力,你自己嘅舊野睇唔明,遺臣士大夫想同普通民眾講自己嘅文化遺產講都講唔到,跟住自己嘅細路全部以普通話同英文去出面搵食,成個共同體形同虛設。
如果粵切字失敗,香港人就一係繼續純漢字粵文而慢性自殺,一係就等拉丁化。而按照現實香港人嘅態度,會繼續純漢字,而光復之後會拉丁化。
個重點係,$_{all natural paths ahead lead to latinisation}$. 如果我哋唔有意識地去扭軚,拉丁化就係我哋嘅命運。用個數學比喻,拉丁化係一個 $_{local minimum}$. 你嘅路徑係趨向嗰度。你入咗去就會好穩定咁喺嗰度發展。

有啲人覺得拉丁化好好。我哋粵字改革學會係唔會駁,亦冇得駁,因為佢哋覺得拉丁化好嘅原因就係繼續用純漢字唔得掂嘅原因。

但係,我哋要嘅,係留畀下一代香港人比我哋呢代好嘅一手牌。拉丁化,係由頭黎過。係要白手興家架。

粵語拉丁化嘅方案好多。大家可以睇睇覺得點,然後諗下,咁樣嘅文字秩序會畀到我哋點樣嘅資產去發展。

\chapter{廣東話喺一眾嘗試逆勢建立自己語言共同體嘅諸漢語言中,有類似周天子嘅崇高地位。}
廣東話喺一眾嘗試逆勢建立自己語言共同體嘅諸漢語言中,有類似周天子嘅崇高地位。如果唔係香港展示過,展示到,同繼續展示緊一種另外嘅語文可能性,顯示住「中文」嘅荒謬,吳語根本就冇可能會有咁樣嘅自我意識甦醒,畀承德話殺死咗淪為《國語》同《詩經》裏面嘅方言之後,仲要畀萬世嘅中文教授言之鑿鑿亂噏廿四,入棺之後墓碑都冇得你正名。


\chapter{粵切字非常之靚}
粵切字並唔樣衰。呢個係客觀事實。乎$_{if}$ 粵切字 $_{is}$ 樣衰,   $_{it is as}$ 樣衰 $_{as}$ 諺文 喃字 新字體。只有腦袋仍然中國中心主義嘅人先至會覺得樣衰。呢點之前講過好多次。基本上任何設計上嘅批評,都係萬民論驢,口爽爽矣。
其餘嘅批評基本上係閱讀理解錯誤嘅結果。「粵語一日用口字旁,就一日係方言」呢句說話嘅意思係,只要粵語係以口字旁黎書寫,文字上同觀感上就必定強烈產生粵語係「次」「附屬」「變體」嘅直覺,繼而有助於嗰啲別有用心去構建「粵語係方言」論述嘅人。鬼唔知粵語唔係方言,問題係個文字系統令人容易話佢係方言。

此外,呢種嘅文字體系的確會令粵語變成為方言 - 語法同詞彙上向主宰住漢字嘅「中文」靠攏 - 亦姐係「藍青化」。呢個現象喺吳語已經非常清楚。我地粵語嘅語法亦已經藍青化非常嚴重。

\chapter{你每日都冇留意嘅粵語漢化}

「Cantonese...may derive from a language similar to proto-Viet-Muong, although a Tai ancestor has also been suggested. In any event, there has been such heavy sinicization that its origins are almost entirely obscured」—— William Mecham

\subsection*{\jcz{粵語 󱝚 漢化 係 樣 發生󱜱?}}
\begin{itemize}
  \item[] 1. 中華主義者隱瞞粵語嘅壯侗語、南亞語源頭,當粵語係漢語族群嘅一種。

  \item[] 2. 虛構「粵語係古漢語/雅言/文言/唐宋口語」嘅傳說,滿足咗粵人嘅虛榮感,粵人就忘記自己真正祖源。
  
  \item[] 3. 忽視越源詞、百越底層,引導粵人以為所有粵語詞都有古華夏源頭。
  
  \item[] 4. 正字運動令人以為所有粵語詞都可以用漢字寫,唔識寫就係你唔夠文化。
\end{itemize}

\subsection*{粵語 漢字化 點解 弊家火}
\begin{itemize}
  \item[] 1. 大批非漢源字詞寫唔出,逐漸被人遺忘。例如liu lun,kal lal,kik lik kak lak。粵文詞語趨向單一死板。
  
  \item[] 2. 新造字吸納唔到入規範書寫系統,例如hea。造字能力受制。
  
  \item[] 3. 非漢源詞假借漢字書寫,同漢字本身意義不吻合,粵文亂晒籠。例如「十蚊雞」,同「蚊」、「雞」根本一啲啦𠹌都冇。
\end{itemize}

\subsection*{一啲誤解與真相}
  
  誤解1
  粵語係漢語嘅一種
  
  真相1
  粵語源頭同泰語、壯語、越南語有關,但漢化程度深過佢哋。
  
  誤解2
  粵語自古以來就係用漢字寫
  
  真相2
  自古以來粵語係冇規範書寫方式,漢字化運動,即係嘗試用漢字寫粵文,歷史不多於二百年。
  
  誤解3
  用漢字寫粵文,係粵文嘅天然形態
  
  真相3
  漢記粵文係粵文嘅一種扭曲形式,唔能夠準確咁紀錄粵語。
  
  誤解4
  漢字拉丁化係共產黨政策
  
  真相4
  中共最落力搞拉丁化,係民國時期佢哋喺佔領區入面做,當中共準備佔領全國嗰陣,拉丁化已經冇乜再提。
  
  誤解5
  用拼音寫粵文即係「漢字拉丁化」
  
  真相5
  拼音粵文係「反對粵文漢字化」,同「漢字拉丁化」一啲關係都冇。
  
  誤解6
  拼音粵文目的係廢除漢字
  
  真相6
  拼音文字可以同漢字並用,好似日文咁。
  
  誤解7
  日本人咁醒,都冇廢除漢字
  
  真相7
  日文嘅漢字使用率不斷下降,其實日本係以自然方式逐漸淘汰漢字,冇一刀切廢除。終有日,日本嘅漢字比例會跌到好似英文嘅拉丁文咁,只會喺學術名詞出現。
  
  誤解8
  拼音=拉丁化
  
  真相8
  拼音粵文重有諺文、注音、假名呢啲選項,甚至用印度化文字亦無不可。
  
  誤解9
  粵語有聲調,唔啱用拼音字
  
  真相9
  世界上大多數聲調語言都係用拼音字,用漢字嘅係屬於標奇立異嘅少數派。
    




\chapter{點解要粵字改革?點解改革要用粵切字?}



% 粵字改革,就係改革粵語所用嘅文字,去完成廣東話嘅書 面語構建工程。粵切字,就係粵字改革嘅最好方案。

% 漢字專用係死路一條。家陣嘅粵文基本上就係漢字專用, 要用漢字呢種阻手阻腳嘅文字黎書寫口語,一係就索性唔 理你口語辭彙,由得佢自生自滅,一係就係靠埋曬啲不成 系統嘅形聲同假借手段。而我地知道,唔寫唔寫,啲 粵詞就會靜々喺黑夜中瓜咗老襯都冇人知。所以,就算 我地而家好似咩都已經可以用漢字寫到,我地冇任何理由 去為此安逸。

% 假借形聲呢種落後當有趣,當醒目嘅並非方法, 係會殺死粵語。粵語漢字專用,面對住呢個又係漢 字專用嘅語言,個博弈格局已經框死咗你,數學上毫無取 勝策略。漢字專用,就好似紮腳咁,箝制住廣東話嘅詞彙 同口頭邏輯嘅自然演化同約定俗成,仲要潛移默化將我哋 嘅語法向普通話靠攏,將我地便成為名符其實嘅方言。最 攞命嘅喺,漢字專用嘅粵文,運作原理同《越人歌》同《萬 葉集》根本就係一樣落後未開發嘅語文,同港女文一樣擔 任唔到文字嘅任務。咁樣嘅語文秩序,喺漢文嘅視覺係唔可能有大義嘅核突化身,喺嘗試學你隻語言嘅外國人黎講 根本就係精神虐待。

% 漢字專用唔得,咁拉丁化呢?至於拉丁化呢家野,係會將 我地嘅文化資本全部一鋪清袋抌曬落鹹水海。更加陰功嘅 係,拉丁化之後,學習粵語就基本上唔會有啲咩價值。你 拉丁化咗,你自己嘅人同英文嘅距離近咗,咁我咁辛苦繼 續講呢種冇咩經濟價值嘅語言做咩?點解唔講英文?你未 答到呢個問題,粵語共同體就會喺你子女嘅嗰一代解 體—— 就好似咗自己子女去國際學校嘅老豆老母,一代 仲係醒目中產香港人,下一代就已經喺 。同樣道理, 啲鬼佬啊盛,見到你隻語言,全部都係拉丁化,唔見得有 啲咩獨特文化。要學嘅,我學你一兩個詞然後喺英文度咪得囉,點解要咁辛苦去學?拉丁化嘅後果,唔係人 人過黎學你。拉丁化咗,廣東話就會變成為英文嘅 大 池,一個用自己嘅血為英文提供源源不絕嘅 嘅世界 二流語言。如果真係拉丁化,就真係囉。

% 粵切字,就係一個考慮曬廣東話書面化路途上所會面對嘅 問題,以最低成本嘗試去奪取最大利益嘅粵文改革方案。

% 粵切字,因為個社會本身巧妙地 同到粵語人嘅 「有邊讀邊」直覺,人人都可以某程度上「無師自通」。

% 亦姐係話,粵切字係可以繞過教育局,喺群眾度散播出去, 成為一種有機文字。加上粵切字可以同漢字混用,粵語文 化自立嘅運動可以完全去中心化,唔使政權都可以食到 糊。

% 粵切字運作簡單,系統易明,所有「有音無字」嘅問題唔 使經啲咩學究過程就可以一次過處理曬。粵切字仲可以畀 廣東話嘅書面語表現到已經同化咗外來詞,同片假名賦予 咗日語吞噬英詞異曲同工。粵切字會核平所有無聊同冇用 嘅「本字考」,高速完成粵文構建工程,將語文發展嘅重 任由士大夫嘅手上轉移至大眾,畀語言嘅主人發揮口語大 民主嘅智慧。長遠而言,以玩文字 為樂嘅士大夫同 文字貴族,粵切字將會慢慢蠶食同凌遲佢地嘅喺文化上來 之不義,用之不利嘅話語權。但唔使驚,佢地墨水所承載 嘅文化唔會消失。佢地所為嘅博大精深文化,會繼續連綿 不斷,但好處係,社會唔需要再為佢地嘅文化自瀆埋單。

% 粵字改革,就係為廣東話放腳。粵切字嘅系統性同可預測 性,將會畀一眾粵語人一個自由、民主、同開放嘅工具去 探索宇宙大義玄黹。粵切字係一種 文字,萬變不 離其宗。粵語終於可以追上泰西文明,唔使下下要討論高 階思哲概念個陣都要本字考到天昏地暗嘈餐懵。更加重要 嘅係,粵語嘅語法特徵唔會再因為漢字專用而被模糊了之。 係「粵漢混用」嘅文字秩序之下,粵語嘅語法特徵唔單止 見得到,我地嘅語言思維更加會因此而變得清晰。漢字嘅 核心,正如老生常談,係象形會意性為主。攞漢字黎寫冇 可能由象形會意性表達嘅語法部件,姐係所為嘅虛詞,根

% 本就唔妥當。粵切字所提供咗嘅一個畀粵語好似日文漢字 假名混用嘅出路,唔單止廢除咗萬惡嘅「口字旁」,仲特 獻埋粵語語法出黎,畀粵語喺視覺上堂堂正正無可否認地 做一直獨立嘅語言。

% 咁樣,戳破語言偽術同濫用語言會易得多。靠思哲博大霧 上位嘅可恥之徒,同埋啲句句歪理喺度污染環境嘅孬種, 將會辭窮而無所遁形。公道唔會因為講噏到嘴唇邊唔出而 要咩「自在人心」自己呃自己,雄辯唔會再喺真理嘅潛反 義詞。思哲不誠實將會成為傾計嘅死罪。更重要嘅係,粵 切字嘅民主化形聲系統,將會畀我地重新用翻我地嘅, 畀我地睇到 ,畀我地喺言談之間言言 。

% 粵切字所應承嘅,就係將粵文化提升至日韓level嘅升le。喺 維持到自己本有文化嘅前提下,粵切字將會畀粵語地位躍 升至日韓地位。除非陰差陽錯搞到好似韓國咁唔少心廢除 咗漢字,粵切字所嘅,係一個畀現時大部分以漢字黎 寫嘅粵文作品繼續延續落去,但又可以擺脫漢字專用遺毒 嘅道路。最緊要嘅喺,粵切字係一個 得好好嘅文字。 就算遞時有人要粵切字專用,廢除漢字,粵切字就好似諺 文咁,係一個隨時隨地都可以畀漢字復活嘅文字系統—— 唔似羅馬化。

% 廣東話嘅書面語運動,唔淨止係五四運動嘅「我手寫我口」。 我地要做到嘅,係為廣東話賦予尊嚴。我地要做到嘅,係講廣東話嘅礻 礻同忄彰顯於世。粵切字要做嘅,就做係 廣東話所配有,所值得擁有嘅文字。粵字改革,就係要為 粵語富裕尊嚴。係因為尊嚴,我地先只要改革,要咁樣改 革。


% \chapter{On the whole, Jyutcitzi is preferable to Jyutping}

% On the whole, Jyutcitzi is preferable to Jyutping
% Cantonese Script Reform Society
% Cantonese Script Reform Society

% ·
% Follow

% 4 min read
% ·
% Sep 22, 2021





% Using Jyutping to teach Cantonese would indeed be extremely helpful for the education and proliferation of Cantonese. However, using Jyutping to accompany the current writing system for Cantonese, is still a very suboptimal solution. Since Cantonese would still be written entirely and only with Chinese characters, which are not phonetic, Jyutping could only play an annotating role, like Hanyu Pinyin for Mandarin. Jyutping, would not be *a* writing system for Cantonese. Jyutping would be used to teach Cantonese, and might be used to annotate Cantonese readings of Chinese characters, but it would not be used as the script in which Cantonese is written. This is entirely like how pinyin is used to annotate Mandarin texts as a reading aid, but the system itself would not be used to write any text. This effectively means that even with Jyutping, as long as Cantonese is written with and only with Chinese characters (sinoglyphs), fluency in Jyutping would not imply any literacy whatsoever. You can know your jyutping very well, but you would still be illiterate if presented a vernacular Cantonese newspaper article written entirely in Sinoglyphs.

% This is why some advocate Cantonese to completely abandon Chinese characters as the script to write Cantonese. Some believe that it is far better to completely romanise Cantonese—— i.e. write Cantonese entirely and only in jyutping. This would be akin to what the Vietnamese and the Zhuang have done.

% This might appear to be the most straightforward and the simplest solution. After all, the logic seems undeniable. The Latin script has time and time again demonstrated its advantages, its flexibility, its impeccable infrastructure in terms of how every single computer on the planet is able to process it without any problems whatsoever. However, the cost of latinisation would be complete decimation and severance of one’s cultural heritage and cultural assets. This is not something to be glossed over. It would mean the decimation of access to old cultural products—— which are used to generate new cultural products, project soft power, and allow the reaping of economic benefits.

% The Cantonese Script Reform Society believes the best way forward, is to adopt a script that is compatible and mixed with the Sinoglyphs—— just like how the Japanese’s writing system allows for the mixing of Kana and Kanji, how the Korean’s allows for the mixed use of Hangul and Hanja. We believe, Jyutcitzi, a phonetic script that is roughly based on the phonetic principle of *faancit*, offers the best way forward.

% Jyutcitzi takes two Chinese characters, and combine them to form one single sound. For example the 廣 gwong2 in 廣東話 is composed of the initial (聲母) gw- and the final (音母) -ong. By combining 古[gw]u and 王w[ong] and composing them, we get one single glyph 古王. Tones would be optionally indicated by means of dakuten-like tone marks, (which are also like the tone marks in the bopomofo system that the Taiwanese use). In essence, we have created a phonetic system, in which Chinese characters would serve as phonetic letters. In particular, we have carefully selected the list of letters such that their spatial combination would yield the best aesthetics.

% Jyutcitzi could also be combined with semantic components—— just like 90% of all Chinese characters are phono-semantic characters, i.e. they are composed of a phonetic component (which roughly suggests the sound) and a semantic component (which roughly suggests the meaning). This means that Jyutcitzi is entirely in line with the evolutionary pathway of Chinese characters, and indeed with how Cantonese speakers have long been inventing new characters. In fact, this principle is not new at all. This principle of cleverly using Chinese characters to write one’s language has long been used by Hakka characters, Chu Nom (the Vietnamese characters), Zhuang characters, and even in certain Korean Gukja and Japanese Kanji as well. Certain script reform proposals from Japan in the late 19th century are exactly like this.

% Most important of all, because of the 有邊讀邊 intuition (“pronounce by reading the phonetic component if there is one”) Cantonese speakers, especially Hong Kong Cantonese speakers have, this system, if adopted as *the* Cantonese script, could very well proliferate naturally and organically, without need of centralised education authority. Furthermore, given how this script can be used along Chinese characters, this script can seamlessly integrate into current Cantonese writing, thereby maximising cultural continuity and minimising cultural destruction. Most important of all, this script is phonetic, so it carries all the benefits of Jyutping, and that users of this script would be fully functionally literate.

% And there is much left to be said about this script’s cultural potential and its capability in resolving the problems of 有音無字 (incidences in which there is no Chinese character for some spoken Cantonese word) once and for all, with a completely predictable, scalable, and logical system.

% It is our vision to make Cantonese as dignified and as culturally and economically productive as Korean and Japanese. If our vision interests you, we would be very happy to answer your questions and tell you more about our project. We have already created an input system for this script, usable offline on microsoft docs. We are also working hard to bring this online so our reform can take off as speedily as we can. Cantonese deserves a script—— because it deserves dignity.

% PS: in the picture included in the link, the tone marks are represented through soochow numerals. This is an old version. The most updated version now uses bopomofo / dakuten tone marks to indicate tones.

% % https://www.facebook.com/permalink.php?story_fbid=142237560779560&id=100970604906256


% \chapter{口字旁之弊}
% 口字旁之弊,一目瞭然。
% 你可能而家仲未熟粵切字。
% 但係,你嘅子女會。
% 你嘅子女會覺得,點解你地可以頂得順呢個咁恐怖咁不成系統嘅系統—真係神奇。你地嘅腦袋係用咩做嘅呢?
% 你嘅子女,會覺得你可以用到咁樣嘅文字系統,係咪個個有自虐癖好。仲要學校冇教,個個無師自通,好犀利,但係佢地一啲都唔羨慕。佢哋用廣東話黎研究天體物理學之外就冇晒時間去度邊個詞用邊隻字,做埋晒呢啲咁米缸數米嘅野。
% 最重要嘅係,你嘅子女,會為你選擇咗畀佢哋一個更加有理則、可預測、民主、可延伸、有尊嚴嘅文字,而對你感恩。
% 世世代代講廣東話嘅人,都會歌頌你嘅選擇。


% \chapter{convinced}
% I was beginning to be convinced but now I am utterly convinced. That Cantonese must have spaces, like Korean. The calligraphic issue must give way. For the space itself is a grammatical marker that marks the beginning and the end of a word. This tool of demarcation will allow poet and playwright to invent new words by putting words together within the confinements delineated by the spaces between words. Written Cantonese needs all the tools imaginable for it to revitalise and resurrect its lost vocabulary. A Hebrew -esque recycling off ancient words for purposes anew is the way to go. But we can’t do that if we can’t tell if this is a new word because we can’t tell if these  characters familiar so and so sequenced are merely a fanciful poetic playful arrangement or other mark of the invention of a new word, where a familiar noun is turned into a verb or verb is turned into an adjective or an adjective is now henceforth interpreted as a noun in this particular context. 

% 我啱啱開始被說服,但而家完全信服。粵語書寫必須要有空格,正如韓文咁。書法嘅問題必須讓步,因為空格本身就係一個語法標記,標示一個詞語嘅開始同結尾。呢個劃分嘅工具可以畀詩人同劇作家透過將詞語拼埋一齊,喺詞語之間嘅空格劃定嘅界線內,去發明新詞。書面粵語需要所有可以想像嘅工具,去振興同復活失落嘅詞彙。好似希伯來文咁,將古詞回收再利用,用於全新嘅目的。但如果我哋唔能夠知道呢啲字係咪新詞,我哋無法分辨呢啲熟悉嘅字符排列係純粹玩味嘅詩意遊戲,定係創造新詞嘅標記,喺呢個特定嘅語境入面,名詞變成動詞,或者動詞變成形容詞,又或者形容詞從此變成名詞。







\printindex % Print the index










\end{document}
